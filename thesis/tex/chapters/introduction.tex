\chapter{Introduction}
\label{chap:introduction}

Probabilistic and statistical modelling arises in a wide variety of situations. Often, the measurements one uses to perform inference have been corrupted by an unknown error. In addition, one may not have access to a correct model for the particular situation --- the ``true'' model is either unknown, or even impossible to formulate.

In the former case, one naturally assumes a random error associated with the observations, and attempts to infer an unknown parameter from the data while accounting for this randomness. The inference may take the form of a point estimate, confidence region, hypothesis test, etc.

In the latter case, one has no choice but to work with a given, although possibly simplified model, either because of insufficient domain knowledge, or for computational reasons. Some degree of uncertainty about the parameters of such a model is then introduced. It is often beneficial to think of these parameters as random variables themselves, in accordance with the Bayesian methodology \citep{bayes}. Such formulation allows to quantify one's prior beliefs about the parameter values, and then to update them upon receiving new observations.

In this thesis, we work with state-space models (SSMs) consisting of a sequence of observed random variables $\by_t$ indexed by discrete time $t = 1, \ldots, T$, which are assumed to be generated by a latent random process $\bx_t$. The distribution of $\bx_t$ and $\by_t$ is assumed to be parameterized by a static parameter $\btheta$. Our goal is to perform posterior inference about this parameter, given the observed sequence $\left\{\by_t\right\}_{t=1}^T$. Furthermore, we assume that the likelihood function of the SSM is intractable and must be estimated. This assumption is well-grounded, as the likelihood is only available in severely restricted cases to be discussed in \autoref{chap:inference}, together with a formal definition of the SSM.

The contribution is twofold. First, we show how to apply the Approximate Bayesian Computation (ABC) methodology \citep{abc-old-old, abc-old} to obtain an estimate of the likelihood even under a misspecified model for the observed variables $\by_t$. Second, we use our results to model the genetic auto-regulation process in prokaryotes. In such a problem, the observation model is typically misspecified, as all attempts to describe such a complex system are necessarily simplified. The quote by the famous statistician George E. P. Box, \emph{``all models are wrong, but some are useful''} \citep{box-quote}, comes to mind here.

The rest of the thesis is organized as follows. In \autoref{chap:related-work}, we review some of the related work. Literature on Markov Chain Monte Carlo (MCMC) methods is discussed, as well as their use in estimating the parameters of an SSM. We list several results dealing with inference in SSMs with intractable likelihoods, as these are relevant to this thesis. Literature on ABC methods is reviewed as well, along with papers describing how these could be applied to SSMs. Finally, we discuss the application of SSMs to bioinformatics, focusing on molecular biology.

In \autoref{chap:inference}, we properly define the assumed form of a state-space model. We show how one would implement a sampler to approximately infer the static parameters given a sequence of observations. We also show that in this basic form, such sampler is unusable, since it relies on the evaluation of the likelihood function, which is intractable (up to certain special cases). We then describe how this likelihood can be estimated using the particle filter \citep{particle-filter} without affecting the asymptotic properties of the sampler.

\autoref{chap:abc} provides a description of the ABC framework, and how it can be applied to estimate the likelihood even under a misspecified observation model. We discuss the pros and cons of such approach compared to the particle filter described in \autoref{chap:inference}.

\autoref{chap:applications} provides numerical studies, where we apply the model developed in \autoref{chap:abc} to several examples and compare it with the model utilizing the particle filter. This chapter also includes the prokaryotic auto-regulation study discussed above.

Finally, \autoref{chap:conclusion} concludes the thesis and discusses some possible directions to be investigated in the future.
