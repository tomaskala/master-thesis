\chapter{Introduction}
\label{chap:introduction}

Probabilistic modelling arises in a wide variety of situations. Often, the measurements one uses to perform inference have been carried out with an unknown error. Frequently, one also does not have access to a correct model for the particular situation --- the true model is either unknown, or such model is even impossible to formulate.

In the former case, one naturally assumes a random error associated with the observations, and attempts to infer something from the data while accounting for this randomness.

In the latter case, one has no choice but to work with a given, although possibly simplified model, purely because of insufficient domain knowledge. Connected with such a model is some degree of uncertainty about its parameters. It is often beneficial to think of these parameters as random variables themselves, in accordance with the Bayesian methodology \citep{bayes}. Such formulation allows to formulate one's prior beliefs about the parameter values, and then updating them upon receiving new observations.

In this thesis, we work with state-space models (SSMs) consisting of a sequence of observed random variables $y_t$ indexed by discrete time $t = 1, \ldots, T$, which have been generated by a latent random process $x_t,\ t = 1, \ldots, T$. The distribution of $x_t$ and $y_t$ is assumed to be parameterized by a static parameter $\theta$. Our goal is to perform posterior inference about this parameter, given a sequence of observations $y_t$. Furthermore, we assume that the likelihood function of the SSM is intractable and must be approximated. This assumption is well-grounded, as the likelihood is only available in severely restricted cases, to be discussed in \autoref{chap:inference}, together with a formal definition of the SSM.

Our contribution is twofold. First, we show how to apply the Approximate Bayesian Computation (ABC) methodology \citep{abc-old-old, abc-old} to obtain an estimate of the likelihood even under a misspecified observation model. Our formulation allows for arbitrary kernel functions with automatically determined widths, unlike the simple accept-reject routine typically discussed in the literature. Second, we apply the resulting model to the genetic auto-regulation process in prokaryotes. Such situation is suitable for a state-space model with a possibly misspecified observation model, as all attempts to model such a complex system are necessarily simplified. To quote the famous statistician George E. P. Box, \emph{``all models are wrong, but some are useful''} \citep{box-quote}. This statement is particularly true for such situations, and it is our hope that our model is indeed useful.

The rest of the thesis is organized as follows:

In \autoref{chap:related-work}, we discuss the related work. First, we review some classical works on Markov Chain Monte Carlo (MCMC) methods. We also discuss how these can be applied to state-space models with an intractable likelihood function. Next, we overview the literature on ABC methods and how these could be used to obtain a suitable likelihood estimate even when the observation model is incorrect. Finally, we discuss the application of related methods in bioinformatics and molecular biology.

In \autoref{chap:inference}, we describe the assumed form of a state-space model. We show how one would implement a sampler to approximately infer the static parameters given a sequence of observations. We also show that in this basic form, such sampler is unusable, since it relies on the evaluation of the likelihood function of the observed sequence, which is intractable (up to certain special cases). We then describe how this likelihood can be estimated using the particle filter \citep{particle-filter} without affecting the asymptotical properties of the sampler.

\autoref{chap:abc} provides a description of the ABC method, and also how it can be applied to estimate the likelihood even under a misspecified model. We discuss the pros and cons of such approach as well as potential issues and how to address them.

\autoref{chap:applications} provides numerical studies, where we apply the model developed in \autoref{chap:abc} to several examples and compare it with the model utilizing the particle filter. This chapter also includes the prokaryotic auto-regulation study discussed earlier.

Finally, \autoref{chap:conclusion} concludes the thesis and discusses some possible directions to be investigated in the future.
