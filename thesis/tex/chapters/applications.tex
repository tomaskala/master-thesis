\chapter{Applications}
\label{chap:applications}

In this chapter, we apply the models developed earlier to two selected problems. The first problem is the Lotka-Volterra model described in \autoref{sec:lotka-volterra}, the second is a simple model for auto-regulation in prokaryotes considered in \autoref{sec:autoregulation}. The particle filter-based approach is compared with the one depending on ABC methods in various settings and model misspecifications.

Both problems require simulating reactions in order to propagate the system states through time. This is done with the help of the Gillespie algorithm, which is first described in \autoref{sec:gillespie}.

\section{Preliminary: the Gillespie algorithm} \label{sec:gillespie}
The Gillespie algorithm \citep{gillespie1, gillespie2} is used to simulate a stochastic process describing the time evolution of a system of reactions. The discussion given here follows \cite{wilkinson-book}.

\paragraph{Time evolution of a reaction system}
Consider a system consisting of $u$ species $\mathcal{X}_1, \ldots, \mathcal{X}_u$ and $v$ reactions $\mathcal{R}_1, \ldots, \mathcal{R}_v$. The species can describe literal animal species, as is the case in the Lotka-Volterra model in \autoref{sec:lotka-volterra}, or individual molecule types, as in \autoref{sec:autoregulation}. The reactions describe the interactions between these species through time.

Let the number of molecules (or individuals, in case of animal species) of the species $\mathcal{X}_i$ at time $t$ be denoted by $X_{i,t}$, and let $\bm{X}_t = \left(X_{1,t}, \ldots, X_{u,t}\right)^T$. Additionally, let the number of reactions of type $\mathcal{R}_i$ which occurred in a time window $(0, t]$ be denoted by $R_{i,t}$, and let $\bm{R}_t = \left(R_{1,t}, \ldots,R_{v,t}\right)^T$. The evolution of the system from time 0 to time $t$ is described by the equation
\begin{equation} \label{eq:system-evolution}
\bm{X}_t - \bm{X}_0 = \mathbb{S}\bm{R}_t,
\end{equation}
where $\mathbb{S} \in \R^{u \times v}$ is called the stoichiometry matrix of the system, and describes the difference in the number of molecules of each species after each reaction occurs. To gain insight into the meaning of $\mathbb{S}$, it is instructive to write it as
\begin{equation*}
\mathbb{S} = \mathbb{P}_\text{post} - \mathbb{P}_\text{pre},
\end{equation*}
where the element $(i,j)$ of $\mathbb{P}_\text{pre}$ denotes the number of molecules of $\mathcal{X}_i$ before a reaction of type $\mathcal{R}_j$ takes place, and the element $(i,j)$ of $\mathbb{P}_\text{post}$ describes the same quantity \emph{after} it takes place. Equation \eqref{eq:system-evolution} can then be written as
\begin{equation*}
\bm{X}_t = \bm{X}_0 + \left(\mathbb{P}_\text{post} - \mathbb{P}_\text{pre}\right) \bm{R}_t,
\end{equation*}
and describes the net gain in the number of molecules of each species, given their initial numbers, and accounting for their increase/decrease when a number of reactions of each type occurs.

In addition, each reaction $\mathcal{R}_i$ has a stochastic rate constant $c_i$ and a rate law (also called the hazard function) $h_i(\bm{X}_t, c_i)$ associated with it. The interpretation of the hazard function is such that $h_i(\bm{X}_t, c_i) \dx{t}$ is the probability of a reaction of type $\mathcal{R}_i$ occurring in a time interval $(t, t + \dx{t}]$, conditionally on the system being in state $\bm{X}_t$. Such a situation is described by an exponential distribution -- the time to the event of a reaction of type $\mathcal{R}_i$ occurring, assuming no other reaction is taking place, is distributed according to ${\mathcal{E}\mathit{xp}(h_i(\bm{X}_t, c_i))}$. This is however a convenient simplification, since multiple reactions are typically occurring at the same time.

\paragraph{The Gillespie algorithm}
In a system with $v$ reactions and their hazard functions $h_i(\bm{X}_t, c_i)$, the hazard of \emph{some} reaction occurring is
\begin{equation*}
h_0(\bm{X}_t, \bm{c}) = \sum_{i=0}^v h_i(\bm{X}_t, c_i),
\end{equation*}
where $\bm{c} = \left(c_1, \ldots, c_v\right)$. The time to the next reaction is then distributed according to $\mathcal{E}\mathit{xp}(h_0(\bm{X}_t, \bm{c}))$. The particular reaction type is a random variable with a categorical distribution $\mathcal{C}\mathit{at}(\widetilde{h}_1(\bm{X}_t,c_1), \ldots, \widetilde{h}_v(\bm{X}_t,c_v))$, where $\displaystyle \widetilde{h}_i(\bm{X}_t,c_i) = \frac{h_i(\bm{X}_t,c_i)}{h_0(\bm{X}_t, \bm{c})}$.

With the above in mind, the Gillespie algorithm can now be formulated, and is given in \autoref{alg:gillespie}. Its purpose is to simulate the state evolution \eqref{eq:system-evolution} for a given time horizon $T$ while accounting for the randomness in the time until a reaction of a particular type takes place. For the purpose of this algorithm, denote the columns of the stoichiometry matrix $\mathbb{S}$ by $\bm{S}^i, \quad i = 1, \ldots, v$.
\begin{algorithm}[ht]
    \caption{Gillespie algorithm}
    \label{alg:gillespie}
    \begin{algorithmic}[1]
        \Input $\text{Time horizon } T, \text{ rate constants } \bm{c} = \left(c_1, \ldots, c_v\right),\ \newline \text{initial molecule numbers } \bm{X}_0 = \left(X_{1,0}, \ldots, X_{u,0} \right).$
        
        \State $t \gets 0$
        
        \State $\bm{X}_t \gets \bm{X}_0$
        
        \While{$t \leq T$}
            \State $\text{Calculate } h_i(\bm{X}_t, c_i), \quad i = 1, \ldots, v.$
            \State $h_0(\bm{X}_t, \bm{c}) \gets \sum_{i=1}^v h_i(\bm{X}_t, c_i)$
            \State $\text{Calculate } \displaystyle \widetilde{h}_i(\bm{X}_t,c_i) = \frac{h_i(\bm{X}_t,c_i)}{h_0(\bm{X}_t, \bm{c})}, \quad i = 1, \ldots, v.$
            \State $\text{Sample } \dx{t} \sim \mathcal{E}\mathit{xp}(h_0(\bm{X}_t, \bm{c})).$ \Comment{Simulate the time to the next reaction.}
            \State $\text{Sample } j \sim \mathcal{C}\mathit{at}(\widetilde{h}_1(\bm{X}_t,c_1), \ldots, \widetilde{h}_v(\bm{X}_t,c_v)).$ \Comment{Simulate the reaction type.}
            \State $\bm{X}_{t + \dx{t}} \gets \bm{X}_t + \bm{S}^j$ \Comment{Update the state according to the reaction $j$.}
            \State $t \gets t + \dx{t}$
        \EndWhile
        
        \Output $\text{Final state } \bm{X}_t, \text {final time } t.$
    \end{algorithmic}
\end{algorithm}
The algorithm us usually the bottleneck of most simulations, and must be implemented carefully. Otherwise, the simulation becomes unacceptably slow. The final time $t$ is output as well, since it may exceed the horizon $T$. If the Gillespie algorithm is run consecutively during a simulation, the interest is to follow the previous run by starting at its final time $t$.

\section{Lotka-Volterra model} \label{sec:lotka-volterra}

\section{Prokaryotic auto-regulation model} \label{sec:autoregulation}