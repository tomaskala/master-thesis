\chapter{Conclusion and future work}
\label{chap:conclusion}

This thesis deals with static parameter inference in state-space models (SSMs). We approach the problem using the probabilistically consistent and versatile Bayesian framework. This involves formulating a prior density and inferring a posterior distribution of the static parameter given an observed sequence. The inference process is complicated by the intractable likelihood of the state-space model, which prevents the application of standard Bayesian methods.

The state-of-the-art approach to the problem is to approximate the unknown posterior distribution using Markov Chain Monte Carlo (MCMC) sampling from the static parameter space and employing a ``nested'' sequential Monte Carlo filter -- i.e., a particle filter -- for the likelihood evaluation. In particular, this likelihood serves to calculate the MCMC acceptance probability. This is known as the particle Markov Chain Monte Carlo (PMCMC) algorithm. The particle filter can be proved to preserve the asymptotic properties of the sampler, even though the likelihood is only approximated.

A common drawback of this approach is the assumptions of a fully specified data-generating mechanism in the form of a probability density function. In applications, this assumption is often violated and calls for approximations. In this thesis, a novel method inspired by the recently developed approximate Bayesian computation (ABC) filters is proposed. While it preserves the MCMC part of the algorithm, it replaces the particle filter by an adaptive ABC filter. The likelihood is then approximated by applying this ABC filter to each sample from the static parameter space. Compared to the particle filter, this method does not require the data-generating model to be probabilistic and instead allows for deterministic functions commonly occurring in practice.

The resulting algorithm does not introduce any additional computational complexity over the particle filter. Unlike PMCMC, our method does not collapse under a misspecified observation model of the SSM and remains stable even when the observed sequence is corrupted by a heavy-tailed noise. Under a known data-generating model, our algorithm necessarily performs worse than the particle filter, since it brings an additional level of approximation. The bias of the ABC method is only mild in the simulation study and the results are comparable to the particle filter. It is more notable in the much more complex autoregulation model but arguably, the Gillespie algorithm used to simulate reactions is too simplistic. It is likely that employing a more complex simulator would represent the biological process more faithfully and allow for more precise inference.

In future work, generalizations of the adaptive kernel tuning to multiple dimensions should be considered. The tuning procedure utilized in this thesis has been derived in context of one-dimensional kernels and applied coordinate-wise in a multidimensional setting. As a consequence, the individual observation vector elements are assumed independent. More reliable likelihood estimates and, in turn, closer representation of the static parameter posterior could be obtained by exploiting the observation dependencies.

If applications to more complicated biological systems was of interest, one should study more sophisticated ways of simulating the latent states. The Gillespie algorithm is still a naïve method, and fails to cover the details present in systems as complex as those found in molecular biology.