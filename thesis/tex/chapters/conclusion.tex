\chapter{Conclusion and future work}
\label{chap:conclusion}

This thesis deals with static parameter inference in state-space models (SSMs). We approach the problem using the probabilistically consistent and versatile Bayesian framework. This involves formulating a prior density and inferring a posterior distribution of the static parameter given an observed sequence. The inference process is complicated by the intractable likelihood of the state-space model, which prevents the application of standard Bayesian methods.

The state-of-the-art approach to the problem is to approximate the unknown posterior distribution using Markov Chain Monte Carlo (MCMC) sampling from the static parameter space and employing a ``nested'' sequential Monte Carlo filter -- i.e., a particle filter -- for the evaluation of the likelihood. In particular, this likelihood serves for calculation of the MCMC acceptance probability. This is known as the particle Markov Chain Monte Carlo (PMCMC) algorithm. The particle filter can be proved to preserve the asymptotic properties of the sampler, even though the SSM likelihood is only approximated.

A common drawback of this approach is the assumptions of a fully specified data-generating mechanism in the form of a probability density function. In applications, this assumption is often violated and calls for approximations. In this thesis, a novel method inspired by the recently developed approximate Bayesian computation (ABC) filters is proposed. While it preserves the MCMC part of the algorithm, it replaces the particle filter by an adaptive ABC filter. The model likelihood is then approximated by applying this ABC filter to each sample from the static parameter space. Compared to the particle filter, this method does not require the data-generating model to be probabilistic and instead allows for deterministic functions commonly occurring in practice.

{\color{red} TODO: Continue here.}

The particle filter and ABC-based methods were compared on two examples, both requiring the simulation of stochastic reactions using the Gillespie algorithm. The first experiment was the Lotka-Volterra model, the second one a simplified model for prokaryotic auto-regulation. We compared the two approaches in well-specified as well as in misspecified observation model scenarios.

In both cases, the particle filter worked well under the correct model, and completely collapsed under model misspecification. The ABC-based method does not suffer from this collapse, but as an approximation, performs necessarily worse than the particle filter. This is clear especially in the auto-regulation model, which poses a considerably more difficult task than the Lotka-Volterra system.

In future, the possibility to use multidimensional kernel functions could be considered. The kernel width tuning procedure utilized in this thesis has been derived in context of one-dimensional kernels, and applied coordinate-wise in case of multidimensional observations. As a consequence, the individual observation vector elements are assumed independent. By allowing to model them in a multidimensional settings, one could also exploit their dependencies.

If applications to more complicated biological systems was of interest, one should study more sophisticated ways of simulating the latent states. The Gillespie algorithm is still a naïve method, and fails to cover the details present in systems as complex as those found in molecular biology.