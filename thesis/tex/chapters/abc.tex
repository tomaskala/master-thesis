\chapter{Approximate Bayesian Computation}
\label{chap:abc}

This chapter discusses the methodology of Approximate Bayesian Computation (ABC). We first motivate the use of ABC methods in our problem in \autoref{sec:abc-motivation}. Then, in \autoref{sec:abc-general}, we describe the method in general, mention some limitations of the basic formulation, and discuss how to address them using kernel functions. \autoref{sec:abc-ssm} then introduces the ABC to our state-space model framework. Finally, in \autoref{sec:abcmh}, we describe how exactly is the ABC method used in our model, and provide an alternative variant of the Metropolis-Hastings algorithm which utilizes ABC instead of the particle filter to provide a likelihood estimate.


\section{Motivation} \label{sec:abc-motivation}
In the previous chapter, we have derived a way to bypass the likelihood function evaluation when calculating the Metropolis-Hastings acceptance ratio. The method relies on the particle filter to calculate a set of weights $w_t^{(i)} \propto \obs_t(\by_t \mid \bx_t^{(i)}, \btheta)$, where $\obs_t$ is the observation model defined in \eqref{eq:factorization}. These weights are then used to estimate the likelihood $p(\by_{1:T} \mid \btheta)$. However, calculating the weights in such way requires full knowledge of this observation model.

In practice, one may not have access to a correct observation model in the form of a probability distribution $\obs_t$. Instead, only a model of the process which generates an observation $\by_t$ from the latent state $\bx_t$ may be available. This generative process may be described by a differential equation, chemical reaction, simulation, etc. One is then in possession of a mean to generate an observation, but not to evaluate how probable it is. By attempting to fit a probability distribution to this generative model, an error is necessarily introduced. The particle filter weights might not necessarily reflect reality, and would lead to incorrect results when using such misspecified model for $\obs_t$.

Instead, we can utilize our knowledge of the generative process $\bx_t \mapsto \by_t$ to simulate a number of pseudo-observations $\bu_t$, and use them to approximate the likelihood $p(\by_{1:T} \mid \btheta)$. Then, one need not evaluate $\obs_t$, and so inference can proceed even without knowing the observation model. This is exactly the gist of the approximate Bayesian computation methodology, and is discussed in more detail in the next section.


\section{ABC in general} \label{sec:abc-general}
This section describes the approximate Bayesian computation method without assuming the SSM model. Only a brief overview is given, since we do not require the full generality of ABC. How to apply the ABC method to SSMs is then discussed in \autoref{sec:abc-ssm}.

The methodology of ABC dates back to \cite{abc-old-old}, where a procedure to use simulated psedudo-observations to approximate the posterior distribution was first described. Lately, ABC methods have gained popularity in modelling biological processes \citep{abc-old}. A more recent review can be found in \cite{abc-recent}.

\paragraph{Approximate Bayesian Computation}

In its classical formulation, ABC provides a way to approximate an intractable posterior $p(\btheta \mid \by) \propto p(\by \mid \btheta) \pprior(\btheta)$ by introducing an auxiliary variable $\bu$. The posterior approximation is then constructed by integrating over this variable, and considering only values sufficiently close to the true measurement. It takes the form of
\begin{equation} \label{eq:abc-integral}
p^\epsilon(\btheta \mid \by) \propto \int \I_{\A_{\epsilon, \by}}(\bu) p(\bu \mid \btheta) \pprior(\btheta) \; \dx{\bu},
\end{equation}
where $\I_A$ is the indicator function of a set $A$, $\A_{\epsilon, \by} = \left\{\bu \in \R^{d_y} : \rho(\bu, \by) \leq \epsilon \right\}$ and $\rho: \R^{d_y} \times \R^{d_y} \to \R$ is a metric.

The motivation behind \eqref{eq:abc-integral} is that such integral can be approximated by randomly sampling from the likelihood $p(\cdot \mid \btheta)$ without needing to evaluate it. This way, the likelihood can exist only conceptually, and we are able to simulate samples $bu^{(i)}$ from a model reflecting some real process, without considering the underlying probability density.

The hyper-parameter $\epsilon \geq 0$ controls how far can the auxiliary variable $\bu$ be from the true measurement $\by$ for them to be considered similar. Obviously, the smaller $\epsilon$, the better approximation is obtained. However, setting $\epsilon$ too small hinders inference, done typically by sampling (discussed in \autoref{alg:abc-rejection} below).

To avoid the curse of dimensionality in case of high-dimensional observations $\by$, a summary statistic $\bm{s}: \R^{d_y} \to \R^p$ where $1 \leq p < d_y$ is often introduced. Instead of comparing $\rho(\bu, \by) \leq \epsilon$, one then compares $\rho(\bm{s}(\bu), \bm{s}(\by)) \leq \epsilon$ (assuming that the metric has been redefined to $\rho: \R^p \times \R^p \to \R$).

It can be shown that if $\bm{s}$ is a sufficient statistic for the parameter $\btheta$, the probability density $p^\epsilon(\btheta \mid \by)$ converges to $p(\btheta \mid \by)$ as $\epsilon \to 0$ \citep{jasra-time-series}. However, it is typically impossible to find such statistic outside of the exponential family of distributions.

\paragraph{Basic version of the ABC simulation}

We now give a basic variant of approximating $p^\epsilon(\btheta \mid \by)$ by sampling. In the spirit of \eqref{eq:abc-integral}, the algorithm performs rejection sampling by comparing whether a sampled $\bu$ is in $\A_{\epsilon, \by}$ or not.
\begin{algorithm}[ht]
    \caption{ABC Rejection Algorithm}
    \label{alg:abc-rejection}
    \begin{algorithmic}[1]
        \Input $\text{Number of samples } M, \text{ observation } \by, \text{ metric } \rho, \text{ maximum distance } \epsilon.$
        
        \State $i \gets 1$
        
        \While{$i \leq M$}
        \State $\text{Sample } \btheta^\prime \sim \pprior(\cdot).$ \Comment{Sample from the prior.}
        \State $\text{Simulate } \bu \text{ from } p(\cdot \mid \btheta^\prime).$ \Comment{Simulate a pseudo-observation.}
        
        \If {$\rho(\bu, \by) \leq \epsilon$}
        \State $\btheta^{(i)} \gets \btheta^\prime$ \Comment{Accept the proposed sample.}
        \State $i \gets i + 1$
        \EndIf
        \EndWhile
        
        \Output $\text{Accepted samples } \left\{ \btheta^{(1)}, \ldots, \btheta^{(M)} \right\}.$
    \end{algorithmic}
\end{algorithm}

The algorithm iteratively samples parameters $\btheta^\prime$ from the prior, plugs them into the likelihood $p(\cdot \mid \btheta^\prime)$, and simulates pseudo-observations $\bu$. These are then compared to the true measurement $\by$ using the metric $\rho$. If the proposed parameter $\btheta^\prime$ gave rise to a pseudo-observation similar enough to the true $\by$ (i.e. $\bu \in \A_{\epsilon, \by}$), the parameter is kept. The posterior distribution $p(\btheta \mid \by)$ is then given in terms of the accepted samples $\btheta^{(1)}, \ldots, \btheta^{(M)}$ as
\begin{equation*}
p^\epsilon(\btheta \mid \by) = \frac{1}{M} \sum_{i=1}^M \delta_{\btheta^{(i)}}(\btheta).
\end{equation*}

Setting a low value of $\epsilon$ increases the approximation accuracy, at the cost of increased rejection rate. On the other hand, setting $\epsilon$ too large causes the algorithm to accept more often, but leads to simulating pseudo-measurements dissimilar to $\by$ and, in turn, incorrect $\btheta^{(i)}$. Setting a correct value of $\epsilon$ is therefore the main difficulty when using ABC. Several approaches are discussed in \cite{jasra-filtering, jasra-time-series}, and one particular way \citep{dedecius} is used in \autoref{sec:abc-ssm} in the context of SSMs.

There are many improvement to the basic ABC of \autoref{alg:abc-rejection}, discussed for instance in \cite{abc-recent}. However, we aim to use ABC only in the framework of the SSMs, we consider only one relevant to us.

\paragraph{Use of kernel functions}

\section{ABC in SSMs} \label{sec:abc-ssm}


\section{Likelihood estimate through ABC} \label{sec:abcmh}