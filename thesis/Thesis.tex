\documentclass[11pt,twoside,openright]{report}

\usepackage[czech,english]{babel}
\usepackage{lmodern}
\usepackage[T1]{fontenc}
\usepackage{kpfonts}
\usepackage[utf8]{inputenc}
\usepackage[a4paper,width=150mm,top=25mm,bottom=25mm]{geometry}
\usepackage{graphicx}
\graphicspath{ {images/} }

\usepackage{emptypage}
\usepackage{fancyhdr}
\pagestyle{fancy}
\fancyhead{}
\fancyhead[RO,LE]{\leftmark}
\fancyfoot{}
\fancyfoot[C]{\thepage}

\usepackage{epstopdf}

\usepackage[nottoc]{tocbibind}
\usepackage[authoryear,round]{natbib}

\usepackage{float}

\usepackage{datetime}
\usepackage[pdftex,final]{hyperref}

\usepackage{algorithm}
\usepackage{algpseudocode}
\usepackage{amsmath}
\usepackage{amsfonts}
\usepackage{amssymb}
\usepackage{mathtools}
\usepackage{interval}
\usepackage{enumitem}

\usepackage{caption}
\usepackage{booktabs}
\usepackage{color}
%\usepackage{subcaption}
\usepackage{subfig}

\usepackage{multirow}
\usepackage{pdfpages}

\usepackage{tikz}
\usepackage{bm}

\addto\extrasenglish{%
  \renewcommand{\chapterautorefname}{Chapter}%
  \renewcommand{\sectionautorefname}{Section}%
  \renewcommand{\subsectionautorefname}{Subsection}%
}

% Make autoref work for the algorithm environment.
\newcommand{\algorithmautorefname}{Algorithm}

%% Definitions %%

% ForEach loop
\algnewcommand\algorithmicforeach{\textbf{for each}}
\algdef{S}[FOR]{ForEach}[1]{\algorithmicforeach\ #1\ \algorithmicdo}

% Break
\newcommand{\Break}{\State \textbf{break} }

% Input
\algnewcommand\algorithmicinput{\textbf{Input:}}
\algnewcommand\Input{\item[\algorithmicinput]}

% Output
\algnewcommand\algorithmicoutput{\textbf{Output:}}
\algnewcommand\Output{\item[\algorithmicoutput]}

\newcommand{\R}{\mathbb{R}}  % Pretty set of real numbers.
\newcommand{\N}{\mathbb{N}}  % Pretty set of natural numbers.
\newcommand{\E}{\mathbb{E}}  % Pretty expectation.
\newcommand{\I}{\mathbb{I}}  % Pretty indicator function.
\DeclarePairedDelimiter{\ceil}{\lceil}{\rceil}  % The ceiling function.
\DeclarePairedDelimiter{\floor}{\lfloor}{\rfloor}  % The floor function.

\newcommand{\vect}[1]{\bm{\MakeLowercase{#1}}}  % Pretty vectors.
\newcommand{\mat}[1]{\mathbf{#1}}  % Pretty matrices.

\DeclareMathOperator*{\argmin}{argmin}  % Argmin
\DeclareMathOperator*{\argmax}{argmax}  % Argmax


\newcommand{\btheta}{\bm{\theta}}
\newcommand{\bx}{\bm{x}}
\newcommand{\by}{\bm{y}}
\newcommand{\bu}{\bm{u}}
\newcommand{\aux}{z}
\newcommand{\auxjoint}{\pi}
\newcommand{\A}{\mathcal{A}}

\newcommand{\trans}{f}
\newcommand{\obs}{g}
\newcommand{\sprior}{p}
\newcommand{\pprior}{\pi}
\newcommand{\prop}{q}
\newcommand{\dx}[1]{\mathrm{d}{#1}}


%% Title %%
\newcommand*{\myTitle}{\begingroup 
    \centering 
    \vspace*{\baselineskip} 
    
    
    {\large Master Thesis} \\
    \vspace*{\baselineskip}
    {\LARGE Bayesian Parameter Estimation of State-Space Models with Intractable Likelihood}    
    \vspace*{5\baselineskip} 
    
    {\Large Bc. Tom\'{a}\v{s} Kala\par} 
    \scshape
    Supervisor: Ing. Kamil Dedecius, PhD.
    
    \vspace*{1\baselineskip}
    \monthname \ \the\year
    
    \vfill
    
    \includegraphics[width=0.4\textwidth]{lion}
    
    \vspace*{1\baselineskip}
    Department of Computer Science\\
    Faculty of Electrical Engineering\\
    Czech Technical University in Prague\\[\baselineskip]
    
    \endgroup\cleardoublepage}


%% Document %%
\begin{document}
\selectlanguage{english}

% Title page
%\input{tex/titlepage}
\begin{titlepage}
\myTitle
\end{titlepage}

%\cleardoublepage
%\pagenumbering{roman}

% Assignment
\shorthandoff{-}  % Czech babel makes trouble here.
\includepdf[pages=-]{tex/assignment}

% Abstract
\selectlanguage{english}
\cleardoublepage
\begin{abstract}
    State-space models (SSMs) are widely used to formalize partially-observed random processes found e.g. in biology, econometrics and signal processing. Given a sequence of observed variables, the interest is to infer a corresponding sequence of latent states assumed to have generated the observations. This procedure is known as filtering. When the SSM is parameterized by a static parameter in addition to the dynamic states, the inference must target both components. The problem then becomes considerably more complex, and the filters typically do not converge. Unless the SSM is linear and Gaussian, its likelihood is intractable, and straightforward inference of the static parameter is not possible. It has been shown that the particle filter can be used as an unbiased estimator of this likelihood even in non-linear models, but the method requires the SSM observation model to be specified as a probability density function. In applications, one is typically in possession of a means to simulate new observations, but not to evaluate their probabilities. Attempts to fit arbitrary probability distributions to the observations typically lead to the particle filter collapsing. Inspired by the techniques of Approximate Bayesian Computation (ABC), this thesis derives an ABC-based filter, which is able to estimate the likelihood even when the observation model is not probabilistic. The performance of the derived algorithm is first demonstrated on a simulation study. Next, the method is applied to a molecular biology problem describing a simplified prokaryotic auto-regulatory network.
    \newline
    \newline
    \noindent \textbf{Keywords:} State-space model, particle filter, Approximate Bayesian Computation, auto-regulation.
\end{abstract}
\selectlanguage{czech}
\cleardoublepage
\begin{abstract}
    Stavové modely představují široce používaný formalismus pro popis částečně pozorovaných náhodných procesů vyskytujících se např. v biologii, ekonometrii a zpracování signálu. Cílem filtrace je odhadnout sekvenci skrytých stavů, o níž předpokládáme, že vygenerovala sekvenci pozorovaných náhodných veličin. Je-li stavový model navíc parametrizován statickým parametrem, je nutné ho zahrnout v inferenci. Celý proces se tím podstatně zkomplikuje, a filtrační algoritmy typicky nekonvergují. Až na případ lineárního Gaussovského stavového modelu není věrohodnostní funkce dostupná, a inference tak není snadná. Bylo ukázáno, že částicový filtr je možné použít jako nestranný odhad věrohodnosti i v nelineárním modelu. Tento odhad ovšem předpokládá, že model pozorování je dán jako hustota pravděpodobnosti. V aplikacích je typicky k dispozici simulace pozorovaných veličin ze skrytých stavů, ale ne vyhodnocení jejich pravděpodobností. Pokusy o modelování pravděpodobnostního rozdělení těchto pozorování pak často vedou ke kolapsu částicového filtru. Inspirováni technikami Approximate Bayesian Computation (ABC) odvodíme filtr schopný odhadnout věrohodnost i v případech, kdy model pozorování není zadán jako hustota pravděpodobnosti. Vyvinutý algoritmus je nejprve otestován v simulační studii. Následně je aplikován na problém z molekulární biologie, ve kterém se pokusíme modelovat zjednodušený autoregulační systém v prokaryotách.
    \newline
    \newline
    \noindent \textbf{Klíčová slova:} Stavový model, částicový filtr, Approximate Bayesian Computation, autoregulace.
\end{abstract}
\selectlanguage{english}

\chapter*{Author statement for graduate thesis:}

I declare that the presented work was developed independently and that I have listed all sources of information used within it in accordance with the methodical instructions for observing the ethical principles in the preparation of university theses.\\

\noindent Prague, date ........................ \hfill ..........................................
\noindent \begin{flushright}
signature
\end{flushright}
\chapter*{Acknowledgements}

I would like to thank my supervisor Kamil Dedecius for being such a positive person. It was a real pleasure to work with him, and I believe I have learned a great deal. Next, I want to thank my family for keeping me alive and well-fed. Finally, I thank my friends Luk\'a\v{s} and Petr for keeping me sane.

% Table of contents
\tableofcontents

% Chapters
%\cleardoublepage
%\pagenumbering{arabic}

\chapter{Introduction}
\label{chap:introduction}

Probabilistic and statistical modelling arises in a wide variety of situations. Often, the measurements one uses to perform inference have been corrupted by an unknown error. In addition, one may not have access to a correct model for the particular situation --- the ``true'' model is either unknown, or even impossible to formulate.

In the former case, one naturally assumes a random error associated with the observations, and attempts to infer an unknown parameter from the data while accounting for this randomness. The inference may take the form of a point estimate, confidence region, hypothesis test, etc.

In the latter case, one has no choice but to work with a given, although possibly simplified model, either because of insufficient domain knowledge, or for computational reasons. Some degree of uncertainty about the parameters of such a model is then introduced. It is often beneficial to think of these parameters as random variables themselves, in accordance with the Bayesian methodology \citep{bayes}. Such formulation allows to quantify one's prior beliefs about the parameter values, and then to update them upon receiving new observations.

In this thesis, we work with state-space models (SSMs) consisting of a sequence of observed random variables $\by_t$ indexed by discrete time $t = 1, \ldots, T$, which are assumed to be generated by a latent random process $\bx_t$. The distribution of $\bx_t$ and $\by_t$ is assumed to be parameterized by a static parameter $\btheta$. Our goal is to perform posterior inference about this parameter, given the observed sequence $\left\{\by_t\right\}_{t=1}^T$. Furthermore, we assume that the likelihood function of the SSM is intractable and must be estimated. This assumption is well-grounded, as the likelihood is only available in severely restricted cases to be discussed in \autoref{chap:inference}, together with a formal definition of the SSM.

The contribution is twofold. First, we show how to apply the Approximate Bayesian Computation (ABC) methodology \citep{abc-old-old, abc-old} to obtain an estimate of the likelihood even under a misspecified model for the observed variables $\by_t$. Second, we use our results to model the genetic auto-regulation process in prokaryotes. In such a problem, the observation model is typically misspecified, as all attempts to describe such a complex system are necessarily simplified. The quote by the famous statistician George E. P. Box, \emph{``all models are wrong, but some are useful''} \citep{box-quote}, comes to mind here.

The rest of the thesis is organized as follows. In \autoref{chap:related-work}, we review some of the related work. Literature on Markov Chain Monte Carlo (MCMC) methods is discussed, as well as their use in estimating the parameters of an SSM. We list several results dealing with inference in SSMs with intractable likelihoods, as these are relevant to this thesis. Literature on ABC methods is reviewed as well, along with papers describing how these could be applied to SSMs. Finally, we discuss the application of SSMs to bioinformatics, focusing on molecular biology.

In \autoref{chap:inference}, we properly define the assumed form of a state-space model. We show how one would implement a sampler to approximately infer the static parameters given a sequence of observations. We also show that in this basic form, such sampler is unusable, since it relies on the evaluation of the likelihood function, which is intractable (up to certain special cases). We then describe how this likelihood can be estimated using the particle filter \citep{particle-filter} without affecting the asymptotic properties of the sampler.

\autoref{chap:abc} provides a description of the ABC framework, and how it can be applied to estimate the likelihood even under a misspecified observation model. We discuss the pros and cons of such approach compared to the particle filter described in \autoref{chap:inference}.

\autoref{chap:applications} provides numerical studies, where we apply the model developed in \autoref{chap:abc} to several examples and compare it with the model utilizing the particle filter. This chapter also includes the prokaryotic auto-regulation study discussed above.

Finally, \autoref{chap:conclusion} concludes the thesis and discusses some possible directions to be investigated in the future.

\chapter{Related work}
\label{chap:related-work}

In this chapter, we provide a survey of literature relevant to our task. Addressed will be works on the use of Markov Chain Monte Carlo methods for approximate inference, works on approximating the likelihood of state-space models by the particle filter, and on Approximate Bayesian Computation methods. We also provide a section describing the use of the considered models in bioinformatics, focusing on molecular biology and genetics.

\section{Markov Chain Monte Carlo methods}
Markov Chain Monte Carlo (MCMC) can be summarized as algorithms designed to simulate random samples from a distribution of interest, which itself is too complicated to sample directly. Assuming the probability density function of this distribution can be evaluated (at least to a multiplicative constant), MCMC methods work by designing a Markov chain whose stationary distribution is the target one.

An attractive property is that the transition distribution of such chain need not resemble the target distribution even closely, and that the problem is relatively unaffected by the dimensionality of the distribution. The downside is a difficulty to determine convergence --- for how long should a chain be ran in order to approximately reach this stationary distribution. In addition, one typically requires independent samples from the target distribution, which, however, the Markov chain samples are \emph{not}. Typically, one needs to ``thin'' the Markov chain samples by keeping every $n$th one to ensure their approximate independence.

Perhaps the best known MCMC algorithm is the Metropolis algorithm \citep{metropolis}, later improved by \cite{hastings}. Random samples are iteratively generated from the Markov chain transition distribution, called the proposal distribution in this context. Each such sample is then compared with the previous one, and accepted with a certain probability which ensures that the stationary distribution is indeed the target. The go-to reference for Monte Carlo methods is \cite{robert-casella}. A particularly appealing treatment of MCMC methods with applications towards physics and machine learning can be found in \cite{information-theory}.

\section{Parameter inference in state-space models}
Assuming that the state-space model (SSM) takes the form informally stated in \autoref{chap:introduction} and more formally given in \autoref{chap:inference}, if all the parameters of interest are changing in time, that is, the inference is about $x_t$ given $y_t$, one arrives at the task of filtering.

If the transition distribution from state $x_t$ to state $x_{t+1}$ is linear in the states and corrupted by uncorrelated additive noise with mean 0, this task can be solved exactly by the Kalman filter \citep{kalman}. The resulting filter is then optimal with respect to the mean squared error. An especially nice overview of the Kalman filter connecting it with other linear statistical models is \cite{lds}.

Once the state transition becomes non-linear, one can use various generalizations of the Kalman filter, such as the extended Kalman filter, which locally linearizes the transition distribution, or the unscented Kalman filter \citep{ukf}.

In recent years, though, the particle filter \citep{particle-filter} has become the most popular alternative due to its particularly simple implementation, appealing asymptotic properties and the fact that it allows for the transition model to be arbitrarily non-linear. The algorithm uses a relatively small number of random samples to approximate the distribution of $x_t$ given $y_1, \ldots, y_t$ at any given time $t$. Since the particle filter is used later in \autoref{chap:inference}, we postpone a more detailed description there.

On the other hand, if some of the unknown parameters are static, the task becomes more complex. Simply applying MCMC algorithms or other approximations is not possible, as the likelihood function, which is a part of the Metropolis-Hastings algorithm, cannot be evaluated. The paper \cite{andrieu} introduced the idea of using the particle filter to obtain an estimate of the likelihood, which has been shown in \cite{del-moral} to preserve the stationary distribution of the underlying Markov chain. The resulting algorithm is called \emph{Marginal Metropolis-Hastings}. A more recent overview can be found in the tutorial by \cite{schoen}.

\section{Approximate Bayesian Computation}
In its original formulation, the method of Approximate Bayesian Computation (ABC) provides a way to approximate the posterior distribution $p(\theta \mid y) \propto f(y \mid \theta) p(\theta)$, assuming that the prior $p(\cdot)$ is fully known, and that the likelihood $f(\cdot \mid \theta)$ can be sampled from, but not evaluated \citep{abc-old-old, abc-old}. A more recent overview of ABC methods can be found in \cite{abc-recent}.

Briefly, the ABC method works by simulating a sample $\hat{\theta}$ from the prior, substituting it to the likelihood, and generating pseudo-observations $\hat{y}$. These are then compared to the real observations $y$, and if they are ``similar enough'', the sample $\hat{\theta}$ is accepted. Otherwise, it is rejected. The posterior distribution of $\theta$ is then given in terms of these random samples $\hat{\theta}$. This variant is referred to as the accept-reject ABC, for obvious reasons.

In this thesis, we apply the ABC method in place of the particle filter to allow for inference about the static parameter $\theta$ when the likelihood is not available. In addition, the use of ABC allows for a possibly misspecified observation model of the SSM, which is often the case, as one may not possess the necessary domain knowledge or computational power needed for the real model. Such a situation has been considered in \cite{jasra-time-series}, although only through the use of the accept-reject variant given above.

Since accepting a sufficient number of samples may take a long time, an idea is to measure the distance between the true and pseudo-observations through a kernel function. This formulation would not reject any samples --- instead, they would get assigned a lower weight. This is considered for instance in \cite{dedecius}, along with a proposed way to automatically tune the kernel width. How to exactly apply the ABC method to our problem will be addressed in \autoref{chap:abc} in detail.

\section{Applications to molecular biology}
Finally, we review works describing how the framework of SSMs and the parameter inference in those can be applied in the context of bioinformatics, focusing on problems of molecular biology and genetics.

The go-to reference for stochastic modelling in biology is \cite{wilkinson-book}. It contains a broad overview of applications of various probabilistic models to examples from molecular biology and chemistry. Included is a description of the Gillespie algorithm \cite{gillespie1, gillespie2} used to simulate chemical reactions, which we will use in \autoref{chap:applications}.

A recent application of SSMs to molecular biology can be found in \cite{wilkinson}, where the authors use the particle filter to approximate the unknown likelihoods of various biological models. We will implement these examples in \autoref{chap:applications} and compare them with the ABC approximation.

The paper \cite{bio1} views biological networks such as gene regulatory networks or signalling pathways as SSMs, and estimates their parameters. The static parameters of the model are viewed as dynamic states which, however, do not change in time. The unscented Kalman filter is then applied to estimate these ``dynamic'' parameters. Such approach is simple, as it does not require the use of MCMC algorithms, but comes without the appealing asymptotical properties of MCMC inference.

\cite{bio2}, \cite{bio3} and \cite{bio4} proceed in a similar fashion when estimating the parameters of various biochemical networks. The used models are only mildly non-linear, and so the extended Kalman filter is sufficient, again without any asymptotical guarantees of identifying the true parameters, however.

An interesting approach to learning the structure of a gene regulatory network from a gene expression time series can be found in \cite{bio5}. First, the particle filter is applied to learn the hidden states of the network. Once these hidden states are known, the LASSO regression is applied to learn a sparse representation of the regulatory network, since each gene is assumed to interact only with a small number of other genes.
\chapter{Learning the parameters of a state-space model}
\label{chap:inference}

This chapter describes the state-space model (SSM) formulation we are working with. In \autoref{sec:ssm-definition}, we state our assumptions about the individual probability distributions. Then in \autoref{sec:parameter-inference}, we calculate the posterior distribution of the parameters of interest, and show that straightforward inference is not possible. Further on, we derive a sampler to approximate this distribution. By itself, this sampler is unusable, as it requires the evaluation of the model likelihood. To circumvent this, we introduce the particle filter in \autoref{sec:particle-filter}. This section gives the definition and some of the properties of the filter. Later in \autoref{sec:particle-filter-estimate} we show how to use the particle filter to estimate the likelihood, and argue that it does not affect the asymptotic properties of the sampler.

Most of this chapter is based on \cite{andrieu} and \cite{schoen}.



\section{State-Space Model definition} \label{sec:ssm-definition}
The state-space model, often also called the hidden Markov model (HMM) assumes a sequence of latent states $\left\{\bx_t\right\}_{t=0}^\infty \subseteq \R^{d_x}$ following a Markov chain, and a sequence of observed variables $\left\{\by_t\right\}_{t=1}^\infty \subseteq \R^{d_y}$. All involved distributions are parameterized by an unknown static parameter $\btheta \in \Theta \subset \R^d$.

For a fixed time $T \geq 1$, we use the shorthands $\bx_{0:T} = \left\{\bx_t\right\}_{t=0}^T$ and $\by_{1:T} = \left\{\by_t\right\}_{t=1}^T$.

The HMM formulation means that the joint distribution of $\bx_{0:T}$ and $\by_{1:T}$ factorizes, for any $T \geq 1$, into
\begin{equation}\label{eq:factorization}
p(\bx_{0:T}, \by_{1:T} \mid \btheta) = \sprior(\bx_0 \mid \btheta) \prod_{t = 1}^{T} \trans_t(\bx_t \mid \bx_{t-1}, \btheta) \obs_t(\by_t \mid \bx_t, \btheta),
\end{equation}
where $\sprior$ is the prior distribution over the initial state, $\trans_t$ is the transition distribution at time $t$ and $\obs_t$ is the observation model at time $t$.

The factorization \eqref{eq:factorization} can be written more clearly as
\begin{align*}
\bx_0 \mid \btheta & \sim \sprior(\cdot \mid \btheta), \\
\bx_t \mid \bx_{t-1}, \btheta & \sim \trans_t(\cdot \mid \bx_{t-1}, \btheta), \quad t = 1, \ldots, T, \\
\by_t \mid \bx_t, \btheta & \sim \obs_t(\cdot \mid \bx_t, \btheta), \quad t = 1, \ldots, T.
\end{align*}

Finally, in accordance with the Bayesian approach \citep{bayes}, we introduce a prior distribution $\pprior$ over the unknown parameters $\btheta$ quantifying our knowledge about $\btheta$ before observing any data. This allows us to state the full joint distribution
\begin{equation}\label{eq:full-joint}
p(\bx_{0:T}, \by_{1:T}, \btheta) = p(\bx_{0:T}, \by_{1:T} \mid \btheta) \pprior(\btheta).
\end{equation}
The corresponding graphical model is depicted in \autoref{fig:graphical-model}.
\begin{figure}[ht]
    \centering
    \begin{tikzpicture}
    % Style
    \tikzstyle{main}=[circle, minimum size = 10mm, thick, draw =black!80, node distance = 16mm]
    \tikzstyle{connect}=[-latex, thick]
    
    % Nodes X
    \node[main,shape=circle,draw=black](X0) at (1,4) {$\bx_0$};
    \node[main,shape=circle,draw=black](X1) at (3,4) {$\bx_1$};
    \node[main,shape=circle,draw=black](X2) at (5,4) {$\bx_2$};
    \node[](Xdots) at (7,4) {$\ldots$};
    \node[main,shape=circle,draw=black](XT) at (9,4) {$\bx_T$};
    
    % Node theta
    \node[](theta) at (7,2) {$\btheta$};
    
    % Nodes Y
    \node[main,shape=circle,draw=black,fill=black!20](Y1) at (3,0) {$\by_1$};
    \node[main,shape=circle,draw=black,fill=black!20](Y2) at (5,0) {$\by_2$};
    \node[](Ydots) at (7,0) {$\ldots$};
    \node[main,shape=circle,draw=black,fill=black!20](YT) at (9,0) {$\by_T$};

    % Edges XX
    \path [->] (X0) edge[connect] node[left] [above] {$\trans_1$} (X1);
    \path [->] (X1) edge[connect] node[left] [above] {$\trans_2$} (X2);
    \path [->] (X2) edge[connect] node[left] [above] {$\trans_3$} (Xdots);
    \path [->] (Xdots) edge[connect] node[left] [above] {$\trans_T$} (XT);
    
    % Edges XY
    \path [->] (X1) edge[connect] node[left] [left] {$\obs_1$} (Y1);
    \path [->] (X2) edge[connect] node[left] [left] {$\obs_2$} (Y2);
    \path [->] (XT) edge[connect] node[left] [left] {$\obs_T$} (YT);
    
    % Edges theta X
    \path [->] (theta) edge[connect] node[left] {} (X0);
    \path [->] (theta) edge[connect] node[left] {} (X1);
    \path [->] (theta) edge[connect] node[left] {} (X2);
    \path [->] (theta) edge[connect] node[left] {} (XT);
    
    % Edges theta Y
    \path [->] (theta) edge[connect] node[left] {} (Y1);
    \path [->] (theta) edge[connect] node[left] {} (Y2);
    \path [->] (theta) edge[connect] node[left] {} (YT);
    \end{tikzpicture}
    \caption{Graphical model describing the full joint distribution \eqref{eq:full-joint}. The shaded nodes denote the observed variables, white nodes represent the latent variables.}
    \label{fig:graphical-model}
\end{figure}



\section{Parameter inference} \label{sec:parameter-inference}
Given an observed sequence $\by_{1:T}$, Bayesian inference relies on the joint posterior density
\begin{equation}\label{eq:joint-posterior}
p(\btheta, \bx_{0:T} \mid \by_{1:T}) = \underbrace{p(\bx_{0:T} \mid \btheta, \by_{1:T})}_{\text{State inference}} \underbrace{p(\btheta \mid \by_{1:T})}_{\text{Parameter inference}}.
\end{equation}
Our primary interest is to perform inference about the static parameter $\btheta$. From \eqref{eq:joint-posterior}, it is clear that to infer about the hidden states $\bx_{0:T}$, one needs knowledge about $\btheta$, so even if the hidden states are of interest, inference about $\btheta$ is necessary. \autoref{sec:particle-filter-estimate} actually shows how to estimate $\bx_{0:T}$ as a by-product.


\paragraph{Bayesian inference}

To perform Bayesian inference about $\btheta$, we express the posterior of $\btheta$ by applying the Bayes theorem:
\begin{equation} \label{eq:posterior}
p(\btheta \mid \by_{1:T}) = \frac{p(\by_{1:T} \mid \btheta) \pprior(\btheta)}{\int p(\by_{1:T} \mid \btheta) \pprior(\btheta) \; \dx{\btheta}}.
\end{equation}

Evaluating the likelihood $p(\by_{1:T} \mid \btheta)$ requires marginalizing over $\bx_{0:T}$:
\begin{equation*}
p(\by_{1:T} \mid \btheta) = \int p(\bx_{0:T}, \by_{1:T} \mid \btheta) \; \dx{\bx_{0:T}},
\end{equation*}
where $p(\bx_{0:T}, \by_{1:T} \mid \btheta)$ is given in \eqref{eq:factorization}. Unless the SSM is linear and Gaussian, such $d_x(T+1)$-dimensional integral is intractable \citep{andrieu}.


\paragraph{Inference under tractable likelihood assumption}

For the time being, we proceed as if the likelihood was tractable. We derive a sampler for $\btheta$ and note which component cannot be evaluated due to the likelihood being present. \autoref{sec:particle-filter-estimate} then describes the necessary changes to allow circumventing the likelihood evaluation.

Often, the interest is not in the posterior $p(\btheta \mid \by_{1:T})$ itself, but on the expectation of some function $\phi$ w.r.t. this distribution, i.e. on
\begin{equation} \label{eq:posterior-integral}
\E_{p(\btheta \mid \by_{1:T})}[\phi(\btheta)] = \int \phi(\btheta) p(\btheta \mid \by_{1:T}) \; \dx{\btheta}.
\end{equation}
We use the Metropolis-Hastings algorithm \citep{metropolis, hastings} to obtain $M$ samples from $p(\btheta \mid \by_{1:T})$, denoted as $\btheta^{(m)},\ m = 1, \ldots, M$. The integral \eqref{eq:posterior-integral} is then approximated as the arithmetic mean
\begin{equation*}
\frac{1}{M} \sum_{m=1}^M \phi(\btheta^{(m)}).
\end{equation*}
An appealing property of the Metropolis-Hastings algorithm is that such arithmetic mean converges to \eqref{eq:posterior-integral} almost surely \citep{robert-casella}, i.e.
\begin{equation*}
\frac{1}{M} \sum_{m=1}^M \phi(\btheta^{(m)}) \xrightarrow{a.s} \int \phi(\btheta) p(\btheta \mid \by_{1:T}) \; \dx{\btheta},
\end{equation*}
where $\xrightarrow{a.s.}$ denotes almost sure convergence.

Finally, we note that if one is indeed interested in the distribution $p(\btheta \mid \by_{1:T})$ itself, it can be recovered by the empirical distribution
\begin{equation*}
\widehat{p}(\btheta \mid \by_{1:T}) = \frac{1}{M} \sum_{m=1}^M \delta_{\btheta^{(m)}}(\btheta),
\end{equation*}
where $\delta$ denotes the Dirac distribution. This estimate can be additionally smoothed using kernel methods \citep{kernel-smoothing}.


\paragraph{Metropolis-Hastings algorithm}
The Metropolis-Hastings algorithm is described in \autoref{alg:metropolis-hastings}. Although well-known, it is included for comparison with the variant introduced in \autoref{alg:marginal-metropolis-hastings}.

The target distribution is the parameter posterior $p(\btheta \mid \by_{1:T}) \propto p(\by_{1:T} \mid \btheta) \pprior(\btheta)$. In this case, it is not necessary to evaluate the normalizing constant, since it gets cancelled out.

The algorithm further requires a proposal distribution $\prop$. Similarly to the prior $\pprior$, it is problem-dependent, and must be selected carefully.

\begin{algorithm}[ht]
    \caption{Metropolis-Hastings}
    \label{alg:metropolis-hastings}
    \begin{algorithmic}[1]
        \Input $\text{Number of samples } M,\ \left\{\by_1, \ldots, \by_T\right\}$
        
        \State $\text{Initialize } \btheta^{(0)}.$
        
        \For{$m = 1\ \mathbf{to}\ M$}
            \State $\text{Sample } \btheta^\prime \sim \prop(\cdot \mid \btheta^{(m-1)}).$
            \State $\text{Calculate the aceptance probability } $ \begin{equation} \label{eq:acceptance-probability}
            \alpha = \min \left\{1, \frac{p(\by_{1:T} \mid \btheta^\prime) \pprior(\btheta^\prime)}{p(\by_{1:T} \mid \btheta^{(m-1)}) \pprior(\btheta^{(m-1)})} \frac{\prop(\btheta^{(m-1)} \mid \btheta^\prime)}{\prop(\btheta^\prime \mid \btheta^{(m-1)})} \right\}.
            \end{equation}
            \State $\text{Sample } u \sim \mathcal{U}(0,1).$
            \If {$u \leq \alpha$}
                \State $\btheta^{(m)} \gets \btheta^\prime$ \Comment{With probability $\alpha$, accept the proposed sample.}
            \Else
                \State $\btheta^{(m)} \gets \btheta^{(m-1)}$ \Comment{With probability $1 - \alpha$, reject the proposed sample.}
            \EndIf
        \EndFor
        
        \Output $\left\{ \btheta^{(1)}, \ldots, \btheta^{(M)} \right\}$
    \end{algorithmic}
\end{algorithm}

We see that the acceptance probability \eqref{eq:acceptance-probability} cannot be calculated, as it depends on the intractable likelihood $p(\by_{1:T} \mid \btheta)$. In \autoref{sec:particle-filter-estimate}, we give a modified variant of the Metropolis-Hastings algorithm, where the likelihood is approximated using the particle filter. The derivation of this filter is the content of the next section.



\section{The particle filter} \label{sec:particle-filter}
The particle filter \citep{particle-filter} is a method for approximating the filtering distribution $p(\bx_t \mid \by_{1:t}, \btheta)$ using a finite number of samples called particles. The algorithm is also known as sequential Monte Carlo or sequential importance sampling. The latter name sheds some light on how the method works, and it is exactly through importance sampling that the particle filter is derived.

\paragraph{Importance sampling}
Here we briefly review the basic idea behind importance sampling. For a more thorough treatment, the reader is referred to \cite{information-theory} or \cite{robert-casella}.

Consider a situation where the expectation of some function $\phi$ w.r.t. the distribution with density $p$,
\begin{equation} \label{eq:is-expectation}
\Phi \coloneqq \E_{p}[\phi(\bm{X})] = \int \phi(\bm{x}) p(\bm{x}) \; \dx{\bm{x}},
\end{equation}
is of interest. Assume that the integral is analytically intractable, and that one cannot generate samples from $p$ to approximate this expectation. Assume further that the density $p$ can be evaluated, at least up to a multiplicative constant, i.e. that it takes the form
\begin{equation*}
p(\bm{x}) = \frac{p^*(\bm{x})}{Z},
\end{equation*}
where $Z$ is an unknown normalizing constant, and $p^*$ can be evaluated. Such situation frequently arises in Bayesian statistics, where a posterior distribution of interest $p(\btheta \mid \bm{x}) = \frac{p(\bm{x} \mid \btheta) p(\btheta)}{\int p(\bm{x} \mid \btheta) p(\btheta) \; \dx{\btheta}}$ is given in terms of the Bayes theorem. The normalizing constant in the denominator is often unavailable in analytic form. However, the numerator can be evaluated.

Next, we introduce a (typically simpler) distribution with probability density $q(\bm{x}) = \frac{q^*(\bm{x})}{Z_Q}$ such that
\begin{enumerate}
    \item One can sample from $q$;
    \item One can evaluate $q^*$;
    \item $p(\bm{x}) > 0$ implies $q(\bm{x}) > 0$.
\end{enumerate}
The expectation \eqref{eq:is-expectation} can then be written as
\begin{equation*}
\Phi = \int \phi(\bm{x}) \frac{q(\bm{x})}{q(\bm{x})} p(\bm{x}) \; \dx{\bm{x}} = \int \phi(\bm{x}) \underbrace{\frac{p(\bm{x})}{q(\bm{x})}}_{w^*(\bm{x})} q(\bm{x}) \; \dx{\bm{x}} = \E_{q}[\phi(\bm{X}) w^*(\bm{X})],
\end{equation*}
where $w^*(\bm{x})$ are called the importance weights. By defining $w(\bm{x}) = \frac{p^*(\bm{x})}{q^*(\bm{x})}$, $\Phi$ can be approximated by
\begin{equation*}
\Phi \approx \widehat{\Phi} \coloneqq \frac{\sum_{i=1}^N \phi(\bm{x}^{(i)}) w(\bm{x}^{(i)})}{\sum_{i=1}^Nw(\bm{x}^{(i)})}, \quad \bm{x}^{(1)}, \ldots, \bm{x}^{(N)} \stackrel{iid}{\sim} q.
\end{equation*}
We note that by using $w$ instead of $w^*$ and normalizing by the weights sum instead of the sample size $N$, we bypass the evaluation of $Z$ and $Z_Q$, since they cancel out. The importance weights here account for correcting the discrepancy between the distribution $q$ and the true distribution $p$.

The estimator $\widehat{\Phi}$ converges to the true expectation $\Phi$ as $N \to \infty$. However, it is not necessarily unbiased \citep{information-theory}.


\paragraph{Sequential importance sampling (SIS)}
The SIS algorithm uses a weighted set of particles $\left\{\left(\bm{x}_t^{(i)}, w_t^{(i)} \right) : i = 1, \ldots, N \right\}$, to represent the filtering distribution $p(\bm{x}_t \mid \by_{1:t}, \btheta)$. To simplify notation, we write $w_t^{(i)}$ instead of $w_t(\bm{x}^{(i)})$ from now on. The empirical approximation to $p(\bm{x}_t~\mid~\by_{1:t}, \btheta)$ is then
\begin{equation*}
\widehat{p}(\bm{x}_t \mid \by_{1:t}, \btheta) = \frac{\sum_{i=1}^N w_t^{(i)} \delta_{\bm{x}_t^{(i)}}(\bm{x}_t)}{\sum_{i=1}^N w_t^{(i)}}.
\end{equation*}

As the name suggests, the algorithm involves a sequential application of the importance sampling procedure with increasing time $t$.

Returning to the SSM \eqref{sec:ssm-definition}, we consider the posterior distribution of a sequence of states $\bx_{0:t}$ given a sequence of observations $\by_{1:t}$. By application of the Bayes theorem, we obtain the following recursive formula:
\begin{equation*}
\begin{split}
p(\bx_{0:t} \mid \by_{1:t}) & \propto p(\by_t \mid \bx_{0:t}, \by_{1:t-1}) p(\bx_{0:t} \mid \by_{1:t-1}) \\
&= \obs_t(\by_t \mid \bx_t) p(\bx_t \mid \bx_{0:t-1}, \by_{1:t-1}) p(\bx_{0:t-1} \mid \by_{1:t-1}) \\
&= \obs_t(\by_t \mid \bx_t) \trans_t(\bx_t \mid \bx_{t-1}) p(\bx_{0:t-1} \mid \by_{1:t-1}),
\end{split}
\end{equation*}
where the equalities follow from the hidden Markov model independence assumptions. For better clarity, we suppress the static parameter $\btheta$ from the conditioning.

For the target $p(\bx_{0:t} \mid \by_{1:t})$, we introduce the importance sampling distribution $q(\bx_{0:t} \mid \by_{1:t})$ and sample $\bx_{0:t}^{(i)}$ from it. The importance weights are (up to normalization) given by
\begin{equation} \label{eq:weight-recursion1}
\begin{split}
w_t^{(i)} & \propto \frac{p(\bx_{0:t}{(i)} \mid \by_{1:t})}{q(\bx_{0:t}^{(i)} \mid \by_{1:t})} \\
& \propto \frac{\obs_t(\by_t \mid \bx_t^{(i)}) \trans_t(\bx_t^{(i)} \mid \bx_{t-1}^{(i)}) p(\bx_{0:t-1}^{(i)} \mid \by_{1:t-1})}{q(\bx_{0:t}^{(i)} \mid \by_{1:t})}.
\end{split}
\end{equation}
By definition of the conditional probability and the hidden Markov model assumptions, we can write the importance sampling distribution as
\begin{equation*}
q(\bx_{0:t} \mid \by_{1:t}) = q(\bx_t \mid \bx_{0:t-1}, \by_{1:t}) q(\bx_{0:t-1} \mid \by_{1:t-1}).
\end{equation*}
By substituting into \eqref{eq:weight-recursion1}, we obtain the following recursion:
\begin{equation} \label{eq:weight-recursion2}
\begin{split}
w_t^{(i)} & \propto \frac{\obs_t(\by_t \mid \bx_t^{(i)}) \trans_t(\bx_t^{(i)} \mid \bx_{t-1}^{(i)})}{q(\bx_t^{(i)} \mid \bx_{0:t-1}^{(i)}, \by_{1:t})} \frac{p(\bx_{0:t-1}^{(i)} \mid \by_{1:t-1})}{q(\bx_{0:t-1}^{(i)} \mid \by_{1:t-1})} \\
& \propto \frac{\obs_t(\by_t \mid \bx_t^{(i)}) \trans_t(\bx_t^{(i)} \mid \bx_{t-1}^{(i)})}{q(\bx_t^{(i)} \mid \bx_{0:t-1}^{(i)}, \by_{1:t})} w_{t-1}^{(i)}.
\end{split}
\end{equation}
So updating the $i$th weight when transitioning from time $t-1$ to $t$ is a relatively simple task involving only multiplication by the first fraction in \eqref{eq:weight-recursion2}.

The sequential importance sampling algorithm is summarized in \autoref{alg:sis}.
\begin{algorithm}[ht]
    \caption{Sequential Importance Sampling}
    \label{alg:sis}
    \begin{algorithmic}[1]
        \Input $\text{Number of particles } N,\ \text{current parameter value } \btheta,\ \left\{\by_1, \ldots, \by_T\right\}$
        
        \State $\text{Sample } \bx_0^{(i)} \sim \sprior(\cdot \mid \btheta), \quad i = 1, \ldots, N.$ \Comment{Initialize $N$ particles.}
        
        \State $w_0^{(i)} \gets \frac{1}{N}, \quad i = 1, \ldots, N.$ \Comment{Initialize uniform weights.}
        
        \For{$t = 1\ \mathbf{to}\ T$}
            \State $\text{Sample } \bx_t^{(i)} \sim q(\cdot \mid \bx_{0:t-1}^{(i)}, \by_{1:t}, \btheta), \quad i = 1, \ldots, N.$ \Comment{Sample $N$ new particles.}
            \State $\text{Set } w_t^{(i)} \propto \frac{\obs_t(\by_t \mid \bx_t^{(i)}, \btheta) \trans_t(\bx_t^{(i)} \mid \bx_{t-1}^{(i)}, \btheta)}{q(\bx_t^{(i)} \mid \bx_{0:t-1}^{(i)}, \by_{1:t}, \btheta)} w_{t-1}^{(i)}, \quad i = 1, \ldots, N.$ \Comment{Update the weights as per \eqref{eq:weight-recursion2}.}
        \EndFor
    \end{algorithmic}
\end{algorithm}
This by itself is almost the particle filter. There are still two issues to be addressed, though. First, the problem of weight degeneracy discussed in the next paragraph. Second, the choice of the importance sampling distribution $q$ addressed later.


\paragraph{Resampling}
A serious problem preventing the use of the SIS algorithm is that the weights degenerate over time. In each time step, the variance of the weights reduces \citep{particle-filter}. This means that the (normalized) weights always converge to a situation where a single weight is 1 and the others are 0.

To alleviate this, the following resampling step is introduced.
\begin{algorithm}[ht]
    \caption{Multinomial resampling}
    \label{alg:resampling}
    \begin{algorithmic}[1]
        \Input $\text{Importance weights } w_t^{(1)}, \ldots, w_t^{(N)},\ \text{particles } \bx_t^{(1)}, \ldots, \bx_t^{(N)}.$
        
        \State $\widetilde{w}_t^{(i)} \gets \frac{w_t^{(i)}}{\sum_{j=1}^N w_t^{(j)}}, \quad i = 1, \ldots, N.$ \Comment{Normalize the weights.}
        
        \State $\text{Sample } a_i \text{ s.t. } \mathbb{P}(a_i = j) = \widetilde{w}_t^{(j)}, \quad i,j = 1, \ldots, N.$ \Comment{Sample indices with replacement.}
        
        \State $w_t^{(a_i)} \gets \frac{1}{N}, \quad i = 1, \ldots, N.$ \Comment{Reset weights.}
        
        \Output $\text{Resampled particles } \bx_t^{(a_1)}, \ldots, \bx_t^{(a_N)} \text{ and weights } w_t^{(a_1)}, \ldots, w_t^{(a_N)}.$
    \end{algorithmic}
\end{algorithm}

The normalized importance weights are interpreted as a probability vector of a categorical distribution. The particles are then resampled (sampled with replacement) according to this distribution. This effectively selects a population of ``strong individuals'' for the next time step.

The above described procedure is known as multinomial resampling. There are other, more complex, approaches, such as stratified resampling \citep{resampling} which further reduce the variance of the weights, at the cost of added computational complexity.

\paragraph{The particle filter}
The remaining step is the choice of the importance sampling distribution $q$. Obviously, the more similar this distribution is to the target, the closer approximation we obtain.

The particle filter arises when the transition distribution $\trans_t$ is chosen as the importance distribution, that is, when
\begin{equation*}
q(\bx_t \mid \bx_{0:t-1}, \by_{1:t}, \btheta) = \trans_t(\bx_t \mid \bx_{t-1}, \btheta).
\end{equation*}
The importance weights \eqref{eq:weight-recursion2} then simplify into
\begin{equation} \label{eq:weight-recursion3}
w_t^{(i)} \propto \obs_t(\by_t \mid \bx_t^{(i)}) w_{t-1}^{(i)}.
\end{equation}
This form of $q$ is not optimal (the target) distribution. It is however ``close'' to it. The optimal distribution would be $q(\bx_t \mid \bx_{0:t-1}, \by_{1:t}, \btheta) = p(\bx_t \mid \bx_{t-1}, \by_{t}, \btheta)$, due to factorization assumptions of the SSM \eqref{eq:factorization}. Calculating it would involve marginalization over $\bx_t$,
\begin{equation*}
p(\bx_t \mid \bx_{t-1}, \by_{t}, \btheta) = \frac{p(\bx_t, \by_{t} \mid \bx_{t-1}, \btheta)}{\int p(\bx_t, \by_{t} \mid \bx_{t-1}, \btheta) \; \dx{\bx_t}} = \frac{\trans_t(\bx_t \mid \bx_{t-1}, \btheta) \obs_t(\by_t \mid \bx_t, \btheta)}{\int \trans_t(\bx_t \mid \bx_{t-1}, \btheta) \obs_t(\by_t \mid \bx_t, \btheta) \; \dx{\bx_t}},
\end{equation*}
which is generally not possible.

The particle filter is summarized in \autoref{alg:particle-filter}.
\begin{algorithm}[ht]
    \caption{Bootstrap particle filter}
    \label{alg:particle-filter}
    \begin{algorithmic}[1]
        \Input $\text{Number of particles } N,\ \text{current parameter value } \btheta,\ \left\{\by_1, \ldots, \by_T\right\}$
        
        \State $\text{Sample } \bx_0^{(i)} \sim \sprior(\cdot \mid \btheta), \quad i = 1, \ldots, N.$ \Comment{Initialize $N$ particles.}
        
        \State $w_0^{(i)} \gets \frac{1}{N}, \quad i = 1, \ldots, N.$ \Comment{Initialize uniform weights.}
        
        \For{$t = 1\ \mathbf{to}\ T$}
        \State $\text{Sample } \bx_t^{(i)} \sim \trans_t(\bx_t \mid \bx_{t-1}, \btheta), \quad i = 1, \ldots, N.$ \Comment{Sample $N$ new particles.}
        
        \State $\text{Set } w_t^{(i)} \propto \obs_t(\by_t \mid \bx_t^{(i)}, \btheta) w_{t-1}^{(i)}, \quad i = 1, \ldots, N.$ \Comment{Update the weights as per \eqref{eq:weight-recursion3}.}
        
        \State $\text{Resample } \bx_t^{(i)} \text{ and reset } w_t^{(i)} \text{ using \autoref{alg:resampling}}, \quad i = 1, \ldots, N.$
        \EndFor
    \end{algorithmic}
\end{algorithm}
The algorithm is called \emph{bootstrap} particle filter, due to resemblance of the resampling step to the non-parametric bootstrap \citep{bootstrap}.



\section{Using the particle filter to estimate the likelihood} \label{sec:particle-filter-estimate}

As mentioned in \autoref{sec:particle-filter} is typically used to approximate the filtering distribution $p(\bx_t \mid \by_{1:t}, \btheta)$. This will be used to provide a tractable approximation to the likelihood $p(\by_{1:T} \mid \btheta)$ such that it does not affect the limiting distribution of the Markov chain used in the Metropolis-Hastings algorithm. This section describes how it is done, and gives the resulting variant of the Metropolis-Hastings algorithm.

\paragraph{Likelihood estimate in general}
Suppose that we are in possession of an estimator $\widehat{\aux}$ of the likelihood $p(\by_{1:T} \mid \btheta)$. As such, it necessarily depends on $\by_{1:T}$ and $\btheta$. Since we aim to use the particle filter to calculate $\widehat{\aux}$, the estimator also depends on the importance weights. This makes the estimator a random variable with some distribution denoted $\psi(\aux \mid \btheta, \by_{1:T})$. It is not necessary to have this distribution available, as it is later shown to cancel out in the Metropolis-Hastings acceptance ratio.

We now return to our model \eqref{eq:posterior} and introduce $\widehat{\aux}$ as an auxiliary variable, along with our variable of interest $\btheta$. This changes the target distribution from $p(\btheta \mid \by_{1:T})$ to
\begin{equation} \label{eq:psi-joint}
\psi(\btheta, \aux \mid \by_{1:T}) = p(\btheta \mid \by_{1:T}) \psi(\aux \mid \btheta, \by_{1:T}) = \frac{p(\by_{1:T} \mid \btheta) \pprior(\btheta)}{p(\by_{1:T})} \psi(\aux \mid \btheta, \by_{1:T}).
\end{equation}
In theory, we could now construct a Metropolis-Hastings algorithm with $\psi(\btheta, \aux \mid \by_{1:T})$ as the target, instead of $p(\btheta \mid \by_{1:T})$ as was the case in \autoref{alg:metropolis-hastings}. However, this would not solve our problem, since calculating the acceptance ratio still requires the calculation of the likelihood $p(\by_{1:T} \mid \btheta)$, as \eqref{eq:psi-joint} makes clear.

Instead, we define a new target distribution over $(\btheta, \widehat{\aux})$ by replacing the likelihood in \eqref{eq:psi-joint} by its estimate $\widehat{\aux}$:
\begin{equation} \label{eq:aux-joint}
\auxjoint(\btheta, \aux \mid \by_{1:T}) \coloneqq \frac{\aux \pprior(\btheta)}{p(\by_{1:T})} \psi(\aux \mid \btheta, \by_{1:T}).
\end{equation}
There are of course some conditions imposed on $\auxjoint(\btheta, \aux \mid \by_{1:T})$ for it to be useful:
\begin{enumerate}
    \item $\auxjoint(\btheta, \aux \mid \by_{1:T})$ must be non-negative for all $(\btheta, \aux)$;
    \item $\auxjoint(\btheta, \aux \mid \by_{1:T})$ must integrate to 1;
    \item the marginal distribution of $\auxjoint(\btheta, \aux \mid \by_{1:T})$ for $\btheta$ must be the original target $p(\btheta \mid \by_{1:T})$.
\end{enumerate}
The first two conditions simply state that $\auxjoint$ is a valid probability distribution. The third condition ensures that by constructing a Metropolis-Hastings algorithm with $\auxjoint$ as the target, the original target distribution is preserved once the auxiliary variables are marginalized out. All three conditions are satisfied is $\widehat{\aux}$ is a non-negative unbiased estimator of the likelihood $p(\by_{1:T} \mid \btheta)$. This is shown as follows.

\begin{enumerate}[align=left]
    \item Non-negativity of $\auxjoint$ follows from the assumed non-negativity of the estimator $\widehat{\aux}$ and validity of the distributions in \eqref{eq:aux-joint}.
    \item[2, 3.] Assume that $\widehat{\aux}$ is an unbiased estimate of $p(\by_{1:T} \mid \btheta)$, i.e. that $\E_{\psi}[\widehat{\aux}] = p(\by_{1:T} \mid \btheta)$. Consider now the marginal of $\auxjoint$ for $\btheta$:
    \begin{equation} \label{eq:marginal}
    \begin{split}
    \int \auxjoint(\btheta, \aux \mid \by_{1:T})\; \dx{\aux} & = \frac{\pprior(\btheta)}{p(\by_{1:T})} \int \aux \psi(\aux \mid \btheta, \by_{1:T}) \; \dx{\aux} \\
    & = \frac{\pprior(\btheta)}{p(\by_{1:T})} \E_{\psi}[\widehat{\aux}] \\
    & = \frac{\pprior(\btheta)}{p(\by_{1:T})} p(\by_{1:T} \mid \btheta) \\
    & = p(\btheta \mid \by_{1:T}),
    \end{split}
    \end{equation}
    the original target distribution. This satisfies condition 3. For condition 2, we simply integrate \eqref{eq:marginal} w.r.t. $\btheta$, which results in 1 due to $p(\btheta \mid \by_{1:T})$ being a valid probability distribution.
\end{enumerate}

\paragraph{Acceptance ratio computation}
Given the new target distribution $\auxjoint$, we can now construct a Metropolis-Hastings algorithm on the joint space of $(\btheta, \aux)$. This means that the proposal density first samples $\btheta^\prime \sim q(\cdot \mid \btheta^{(m-1)})$, and then $\widehat{\aux}^\prime \sim \psi(\cdot \mid \btheta^\prime, \by_{1:T})$. The acceptance ratio can now be computed as
\begin{equation*}
\begin{split}
\alpha & = \min \left\{1, \frac{\auxjoint(\btheta^\prime, \aux^\prime \mid \by_{1:T})}{\auxjoint(\btheta^{(m-1)}, \aux^{(m-1)} \mid \by_{1:T})} \frac{\prop(\btheta^{(m-1)} \mid \btheta^\prime) \psi(\aux^{(m-1)} \mid \btheta^{(m-1)}, \by_{1:T})}{\prop(\btheta^\prime \mid \btheta^{(m-1)}) \psi(\aux^\prime \mid \btheta^\prime, \by_{1:T})} \right\} \\
& = \min \left\{1, \frac{\aux^\prime \pprior(\btheta^\prime) \psi(\aux^\prime \mid \btheta^\prime, \by_{1:T})}{\aux^{(m-1)} \pprior(\btheta^{(m-1)}) \psi(\aux^{(m-1)} \mid \btheta^{(m-1)}, \by_{1:T})} \frac{\prop(\btheta^{(m-1)} \mid \btheta^\prime) \psi(\aux^{(m-1)} \mid \btheta^{(m-1)}, \by_{1:T})}{\prop(\btheta^\prime \mid \btheta^{(m-1)}) \psi(\aux^\prime \mid \btheta^\prime, \by_{1:T})} \right\} \\
& = \min \left\{1, \frac{\aux^\prime \pprior(\btheta^\prime)}{\aux^{(m-1)} \pprior(\btheta^{(m-1)})} \frac{q(\btheta^{(m-1)} \mid \btheta^\prime)}{q(\btheta^\prime \mid \btheta^{(m-1)})} \right\}.
\end{split}
\end{equation*}

Since \eqref{eq:marginal} shows that the marginal of $\auxjoint$ for $\btheta$ is the original target $p(\btheta \mid \by_{1:T})$, all we need to do is to discard the sampled $\widehat{\aux}^{(m)}$ and keep only $\btheta^{(m)}$ when running Metropolis-Hastings on the joint space of $(\btheta, \aux)$.

\paragraph{Calculating the estimate using the particle filter}
Finally, we describe how exactly is the particle filter used as an estimator of $p(\by_{1:T} \mid \btheta)$.

First, we decompose the likelihood into a product of simpler distributions, which are then marginalized over the corresponding hidden state:
\begin{equation} \label{eq:likelihood-factorization}
\begin{split}
p(\by_{1:T} \mid \btheta) &= \prod_{t=1}^T p(\by_t \mid \by_{1:t-1}, \btheta) \\
&= \prod_{t=1}^T \int p(\by_t, \bx_t \mid \by_{1:t-1}, \btheta) \; \dx{\bx_t} \\
&= \prod_{t=1}^T \int p(\by_t \mid \bx_t, \btheta) p(\bx_t \mid \by_{1:t-1}, \btheta) \; \dx{\bx_t}.
\end{split}
\end{equation}

Using the particles $\left\{\bx_t^{(i)}\right\}_{i=1}^N$, we plug in the empirical approximation to $p(\bx_t \mid \by_{1:t-1}, \btheta)$, $\widehat{p}(\bx_t \mid \by_{1:t-1}, \btheta) = \frac{1}{N} \sum_{i=1}^N \delta_{\bm{x}_t^{(i)}}(\bx_t)$, into \eqref{eq:likelihood-factorization}, obtaining
\begin{equation*}
\begin{split}
p(\by_{1:T} \mid \btheta) & \approx \prod_{t=1}^T \int p(\by_t \mid \bx_t, \btheta) \left[ \frac{1}{N} \sum_{i=1}^N \delta_{\bm{x}_t^{(i)}}(\bx_t) \right] \; \dx{\bx_t} \\
& = \prod_{t=1}^T \frac{1}{N} \sum_{i=1}^N \int p(\by_t \mid \bx_t, \btheta) \delta_{\bm{x}_t^{(i)}}(\bx_t) \; \dx{\bx_t} \\
& = \prod_{t=1}^T \frac{1}{N} \sum_{i=1}^N p(\by_t \mid \bx_t^{(i)}, \btheta)
\end{split}
\end{equation*}
due to linearity of the integral and properties of the Dirac distribution.

In $p(\by_t \mid \bx_t^{(i)}, \btheta)$, we recognize the particle filter weights $w_t^{(i)}$ defined in \eqref{eq:weight-recursion3}. This allows us to finally define the likelihood estimate as
\begin{equation} \label{eq:likelihood-estimate}
\widehat{\aux} \coloneqq \prod_{t=1}^T \frac{1}{N} \sum_{i=1}^N w_t^{(i)}.
\end{equation}
This estimator is obviously non-negative due to construction of the weights. The proof that it is also unbiased (and therefore also integrates to 1) is more involved, and the reader is referred to \cite{del-moral} for the original proof.

Finally, we describe the resulting variant of the Metropolis-Hastings algorithm employing the likelihood estimate \eqref{eq:likelihood-estimate}.

\begin{algorithm}[ht]
    \caption{Marginal Metropolis-Hastings}
    \label{alg:marginal-metropolis-hastings}
    \begin{algorithmic}[1]
        \Input $\text{Number of samples } M,\ \left\{\by_1, \ldots, \by_T\right\}$
        
        \State $\text{Initialize } \btheta^{(0)}.$
        \State $\text{Run \autoref{alg:particle-filter} with } \btheta^{(0)} \text{ to obtain the weights } w_{0,t}^{(i)}, \quad t = 1, \ldots, T,\ i = 1, \ldots, N.$
        \State $\text{Calculate } \widehat{\aux}^{(0)} \text{ according to \eqref{eq:likelihood-estimate} using } w_{0,t}^{(i)}.$
        
        \For{$m = 1\ \mathbf{to}\ M$}
        \State $\text{Sample } \btheta^\prime \sim \prop(\cdot \mid \btheta^{(m-1)}).$
        \State $\text{Run \autoref{alg:particle-filter} with } \btheta^\prime \text{ to obtain the weights } w_{m,t}^{(i)}, \quad t = 1, \ldots, T, \ i = 1, \ldots, N.$
        \State $\text{Calculate } \widehat{\aux}^\prime \text{ according to \eqref{eq:likelihood-estimate} using } w_{m,t}^{(i)}.$
        \State $\text{Calculate the aceptance probability } $ \begin{equation} \label{eq:acceptance-probability-tractable}
        \alpha = \min \left\{1, \frac{\widehat{\aux}^\prime \pprior(\btheta^\prime)}{\widehat{\aux}^{(m-1)} \pprior(\btheta^{(m-1)})} \frac{\prop(\btheta^{(m-1)} \mid \btheta^\prime)}{\prop(\btheta^\prime \mid \btheta^{(m-1)})} \right\}.
        \end{equation}
        \State $\text{Sample } u \sim \mathcal{U}(0,1).$
        \If {$u \leq \alpha$}
        \State $\left( \btheta^{(m)}, \widehat{\aux}^{(m)} \right) \gets \left( \btheta^\prime, \widehat{\aux}^\prime \right)$ \Comment{With probability $\alpha$, accept the proposed sample.}
        \Else
        \State $\left( \btheta^{(m)}, \widehat{\aux}^{(m)} \right) \gets \left( \btheta^{(m-1)}, \widehat{\aux}^{(m-1)} \right)$ \Comment{With probability $1 - \alpha$, reject the proposed sample.}
        \EndIf
        \EndFor
        
        \Output $\left\{ \btheta^{(1)}, \ldots, \btheta^{(M)} \right\}$
    \end{algorithmic}
\end{algorithm}

This algorithm, called marginal Metropolis-Hastings, was introduced in \cite{andrieu}. Compared to \autoref{alg:metropolis-hastings}, all components of this algorithm can be evaluated. Due to construction of the estimator $\widehat{\aux}$, the marginal of the limiting distribution of \autoref{alg:marginal-metropolis-hastings} is the original target $p(\btheta \mid \by_{1:T})$.
\chapter{Approximate Bayesian Computation}
\label{chap:abc}
\chapter{Applications}
\label{chap:applications}

In this chapter, we apply the models developed earlier to two selected problems. The first problem is the Lotka-Volterra model described in \autoref{sec:lotka-volterra}, the second is a simple model for auto-regulation in prokaryotes considered in \autoref{sec:autoregulation}. The particle filter-based approach is compared with the one depending on ABC methods in various settings and model misspecifications.

Both problems require simulating reactions in order to propagate the system states through time. This is done with the help of the Gillespie algorithm, which is first described in \autoref{sec:gillespie}.


\section{Implementation notes}
All of the experiments described below have been implemented in Python 3.6.5. The performance-critical parts were additionally written in Cython to obtain C-like performance. The only additional dependencies are NumPy 1.14.3, SciPy 1.2.1, Matplotlib 2.2.2 and Statsmodels 0.9.0. The experiments have been performed on a standard laptop computer.


\section{Preliminary: the Gillespie algorithm} \label{sec:gillespie}
The Gillespie algorithm \citep{gillespie1, gillespie2} is used to simulate a stochastic process describing the time evolution of a system of reactions. The discussion given here follows \cite{wilkinson-book}.

\paragraph{Time evolution of a reaction system}
Consider a system consisting of $u$ species $\mathcal{X}_1, \ldots, \mathcal{X}_u$ and $v$ reactions $\mathcal{R}_1, \ldots, \mathcal{R}_v$. The species can describe literal animal species, as is the case in the Lotka-Volterra model in \autoref{sec:lotka-volterra}, or individual molecule types, as in \autoref{sec:autoregulation}. The reactions describe the interactions between these species through time.

Let the number of molecules (or individuals, in case of animal species) of the species $\mathcal{X}_i$ at time $t$ be denoted by $X_{i,t}$, and let $\bm{X}_t = \left(X_{1,t}, \ldots, X_{u,t}\right)^\intercal$. Additionally, let the number of reactions of type $\mathcal{R}_i$ which occurred in a time window $(0, t]$ be denoted by $R_{i,t}$, and let $\bm{R}_t = \left(R_{1,t}, \ldots,R_{v,t}\right)^\intercal$. The evolution of the system from time 0 to time $t$ is described by the equation
\begin{equation} \label{eq:system-evolution}
\bm{X}_t - \bm{X}_0 = \mathbb{S}\bm{R}_t,
\end{equation}
where $\mathbb{S} \in \R^{u \times v}$ is called the stoichiometry matrix of the system, and describes the difference in the number of molecules of each species after each reaction occurs. To gain insight into the meaning of $\mathbb{S}$, it is instructive to write it as
\begin{equation*}
\mathbb{S} = \mathbb{P}_\text{post} - \mathbb{P}_\text{pre},
\end{equation*}
where the element $(i,j)$ of $\mathbb{P}_\text{pre}$ denotes the number of molecules of $\mathcal{X}_i$ before a reaction of type $\mathcal{R}_j$ takes place, and the element $(i,j)$ of $\mathbb{P}_\text{post}$ describes the same quantity \emph{after} it takes place. Equation \eqref{eq:system-evolution} can then be written as
\begin{equation*}
\bm{X}_t = \bm{X}_0 + \left(\mathbb{P}_\text{post} - \mathbb{P}_\text{pre}\right) \bm{R}_t,
\end{equation*}
and describes the net gain in the number of molecules of each species, given their initial numbers, and accounting for their increase/decrease when a number of reactions of each type occurs.

In addition, each reaction $\mathcal{R}_i$ has a stochastic rate constant $c_i$ and a rate law (also called the hazard function) $h_i(\bm{X}_t, c_i)$ associated with it. The interpretation of the hazard function is such that $h_i(\bm{X}_t, c_i) \dx{t}$ is the probability of a reaction of type $\mathcal{R}_i$ occurring in a time interval $(t, t + \dx{t}]$, conditionally on the system being in state $\bm{X}_t$. Such a situation is described by an exponential distribution -- the time to the event of a reaction of type $\mathcal{R}_i$ occurring, assuming no other reaction is taking place, is distributed according to ${\mathcal{E}\mathit{xp}\left(h_i(\bm{X}_t, c_i)\right)}$. This is however a convenient simplification, since multiple reactions are typically occurring at the same time.

\paragraph{The Gillespie algorithm}
In a system with $v$ reactions and their hazard functions $h_i(\bm{X}_t, c_i)$, the hazard of \emph{some} reaction occurring is
\begin{equation*}
h_0(\bm{X}_t, \bm{c}) = \sum_{i=0}^v h_i(\bm{X}_t, c_i),
\end{equation*}
where $\bm{c} = \left(c_1, \ldots, c_v\right)^\intercal$. The time to the next reaction is then distributed according to $\mathcal{E}\mathit{xp}\left(h_0(\bm{X}_t, \bm{c})\right)$. The particular reaction type is a random variable with a categorical distribution $\mathcal{C}\mathit{at}\left(\widetilde{h}_1(\bm{X}_t,c_1), \ldots, \widetilde{h}_v(\bm{X}_t,c_v)\right)$, where $\displaystyle \widetilde{h}_i(\bm{X}_t,c_i) = \frac{h_i(\bm{X}_t,c_i)}{h_0(\bm{X}_t, \bm{c})}$.

With the above in mind, the Gillespie algorithm can now be formulated, and is given in \autoref{alg:gillespie}. Its purpose is to simulate the state evolution \eqref{eq:system-evolution} for a given time horizon $T$ while accounting for the randomness in the time until a reaction of a particular type takes place. For the purpose of this algorithm, denote the columns of the stoichiometry matrix $\mathbb{S}$ by $\bm{S}^i, \quad i = 1, \ldots, v$.
\begin{algorithm}[ht]
    \caption{Gillespie algorithm}
    \label{alg:gillespie}
    \begin{algorithmic}[1]
        \Input $\text{Time horizon } T, \text{ rate constants } \bm{c} = \left(c_1, \ldots, c_v\right)^\intercal,\ \text{initial molecule numbers } \bm{X}_0.$
        
        \State $t \gets 0$
        
        \State $\bm{X}_t \gets \bm{X}_0$
        
        \While{$t \leq T$}
            \State $\text{Calculate } h_i(\bm{X}_t, c_i), \quad i = 1, \ldots, v.$
            \State $h_0(\bm{X}_t, \bm{c}) \gets \sum_{i=1}^v h_i(\bm{X}_t, c_i)$
            \State $\text{Calculate } \displaystyle \widetilde{h}_i(\bm{X}_t,c_i) = \frac{h_i(\bm{X}_t,c_i)}{h_0(\bm{X}_t, \bm{c})}, \quad i = 1, \ldots, v.$
            \State $\text{Sample } \dx{t} \sim \mathcal{E}\mathit{xp}\left(h_0(\bm{X}_t, \bm{c})\right).$ \Comment{Simulate the time to the next reaction.}
            \State $\text{Sample } i \sim \mathcal{C}\mathit{at}\left(\widetilde{h}_1(\bm{X}_t,c_1), \ldots, \widetilde{h}_v(\bm{X}_t,c_v)\right).$ \Comment{Simulate the reaction type.}
            \State $\bm{X}_{t + \dx{t}} \gets \bm{X}_t + \bm{S}^i$ \Comment{Update the state according to the reaction $i$.}
            \State $t \gets t + \dx{t}$
        \EndWhile
        
        \Output $\text{Final state } \bm{X}_t, \text { final time } t.$
    \end{algorithmic}
\end{algorithm}

The algorithm is usually the bottleneck of most simulations, and must be implemented carefully; otherwise, the simulation becomes unacceptably slow. The final time $t$ is at the output as well, since it may exceed the horizon $T$. If the algorithm is run consecutively during a simulation, the interest is to follow the previous run by starting at its final time $t$.

\section{Lotka-Volterra model} \label{sec:lotka-volterra}
\subsection{Problem description}
The first considered problem is the Lotka-Volterra model \citep{lotka, volterra}. The system describes a simplified time interaction of a population consisting of a predator and prey species. Denoting the prey species by $\mathcal{X}_1$ and the predator species by $\mathcal{X}_2$, the system can be described by the reactions
\begin{align}
\mathcal{R}_1:\quad & \mathcal{X}_1 \to 2 \mathcal{X}_1, \label{eq:lv1} \\
\mathcal{R}_2:\quad & \mathcal{X}_1 + \mathcal{X}_2 \to 2 \mathcal{X}_2, \label{eq:lv2} \\
\mathcal{R}_3:\quad & \mathcal{X}_2 \to \emptyset. \label{eq:lv3}
\end{align}
Equation \eqref{eq:lv1} describes the reproduction of the prey species. Equation \eqref{eq:lv2} describes the interaction between the predator and the prey where a predator consumes an individual of the prey species and produces an offspring. Equation \eqref{eq:lv3} describes the extinction of the predator species when no prey is present.

The state of the system at time $t$ is $\bm{X}_t = \left(X_{1,t}, X_{2,t}\right)^\intercal$. The stoichiometry matrix is given by
\begin{equation*}
\mathbb{S} = \begin{pmatrix}
1 & -1 & 0 \\
0 & 1 & -1 \\
\end{pmatrix},
\end{equation*}
and the hazard functions vector is $\bm{h}(\bm{X}_t, \bm{c}) = \left(c_1 X_{1,t}, c_2 X_{1,t} X_{2,t}, c_3 X_{2,t}\right)^\intercal$ \citep{wilkinson}. Although simple to describe, this model is analytically intractable \citep{wilkinson-book}.

For the inference problem, we consider the unknown parameters to be $\btheta = \left(c_1, c_2, c_3\right)^\intercal$, and the state at time $t$ to be $\bx_t = \left(X_{1,t}, X_{2,t}\right)^\intercal$. Since the rate constants $c_1, c_2, c_3$ are by definition positive, we are working in the log space to avoid restricting ourselves to positive support distributions. The model is specified by the following:
\begin{equation*}
\begin{split}
\sprior(x_{1,0} \mid \btheta) &= \mathcal{P}\mathit{o}\left(50\right), \\
\sprior(x_{2,0} \mid \btheta) &= \mathcal{P}\mathit{o}\left(100\right), \\
\trans_t(\bx_t \mid \bx_{t-1}, \btheta) & \text{ is simulated using \autoref{alg:gillespie} }, \\
\obs_t(\by_t \mid \bx_{t}, \btheta) &= \mathcal{N}_2\left(\bx_t, 10^2\right).
\end{split}
\end{equation*}
The parameter prior and the Metropolis-Hastings proposal are additionally given by
\begin{equation*}
\begin{split}
\pprior(\log c_i) &= \mathcal{U}\left(-7, 2\right), \quad i = 1, 2, 3, \\
\prop(\btheta^\prime \mid \btheta) &= \mathcal{N}_3\left(\btheta, \text{diag}\left(0.01, 0.01, 0.01\right)\right).
\end{split}
\end{equation*}
The initial parameters are $\btheta^{(0)} = \left(1, 0.005, 0.6\right)^\intercal$. We are working with a dataset simulated using \autoref{alg:gillespie} ran so that it outputs 16 observations $\by_t$, starting from the same parameters $\btheta^{(0)}$. The inference is started in the correct parameters and applied on a short sequence only; its purpose is only to demonstrate that the algorithm is able to identify these parameters.

We apply the marginal Metropolis-Hastings algorithm depending on the particle filter as well as the one utilizing ABC methods. Both algorithms are ran for $M = 50000$ samples with $N = 100$ particles. Additionally, the number of covered pseudo-observations in the ABC formulation is $\alpha = 95$ and the volume of the p-HPR is $p = 0.95$.

When using the ABC method, we simulate pseudo-observations from the observation model $\obs_t$ without the noise term. That is, we simulate $\bu_t = \bx_t$, deterministically.

\subsection{Inference using the particle filter}
In this section, we consider inference using \autoref{alg:marginal-metropolis-hastings}, i.e. using the particle filter to approximate the likelihood.

\paragraph{Correctly specified observation model}
At first, we assume the correct model specification. This means that the observations $\by_t$ have been corrupted by Gaussian noise, and the observation model $\obs_t$ is Gaussian as well. In this situation, the particle filter is expected to perform well, as all assumptions have been met.

In \autoref{fig:lv-pmh-gauss-c1}, \autoref{fig:lv-pmh-gauss-c2} and \autoref{fig:lv-pmh-gauss-c3}, the results for the parameters $c_1$, $c_2$ and $c_3$, respectively, are shown. For each parameter, we show (in this order) the trace plot, the autocorrelation plot, and the histogram of sampled values. We use no burn-in period, as the inference starts in the correct values, but apply thinning of 100, i.e. keep every 100th sample. This ensures relatively uncorrelated samples, as is clear from the auto-correlation plots. The acceptation rate of the Metropolis-Hastings algorithms moves around 20 \%.

\begin{figure}[ht]
    \centering
    \includegraphics[width=0.7\linewidth]{lotka-volterra/pmh_gauss_c1}
    \caption{Particle filter-based inference of the parameter $c_1$ in the Lotka-Volterra model. Uses Gaussian noise and a Gaussian observation model. The true value is shown in red.}
    \label{fig:lv-pmh-gauss-c1}
\end{figure}

\begin{figure}[ht]
    \centering
    \includegraphics[width=0.7\linewidth]{lotka-volterra/pmh_gauss_c2}
    \caption{Particle filter-based inference of the parameter $c_2$ in the Lotka-Volterra model. Uses Gaussian noise and a Gaussian observation model. The true value is shown in red.}
    \label{fig:lv-pmh-gauss-c2}
\end{figure}

\begin{figure}[ht]
    \centering
    \includegraphics[width=0.7\linewidth]{lotka-volterra/pmh_gauss_c3}
    \caption{Particle filter-based inference of the parameter $c_3$ in the Lotka-Volterra model. Uses Gaussian noise and a Gaussian observation model. The true value is shown in red.}
    \label{fig:lv-pmh-gauss-c3}
\end{figure}

In all cases, the correct parameters are well-covered by the sampled values, while allowing for some degree of variance. The histograms provide estimated posterior distributions of the individual parameters. The sampled values can be used to provide point estimates or credible intervals for the true parameters.

\paragraph{Misspecified observation model}
Next, we keep the Gaussian observation model, but corrupt the observation sequence by a Cauchy noise with scale 10. Arguably, this scale is quite high, but is used to match the scale of the Gaussian noise from the previous section. The heavy-tailed Cauchy distribution allows sampling distant noise terms, and corrupts the observation sequence $\by_t$ much more severely. The Gaussian observation model $\obs_t$ then assigns probability close to zero to these values, and the filter collapses. This is clear from \autoref{fig:lv-pmh-cauchy-c1}, \autoref{fig:lv-pmh-cauchy-c2} and \autoref{fig:lv-pmh-cauchy-c3}.

\begin{figure}[ht]
    \centering
    \includegraphics[width=0.7\linewidth]{lotka-volterra/pmh_cauchy_c1}
    \caption{Particle filter-based inference of the parameter $c_1$ in the Lotka-Volterra model. Uses Cauchy noise and a Gaussian observation model. The true value is shown in red.}
    \label{fig:lv-pmh-cauchy-c1}
\end{figure}

\begin{figure}[ht]
    \centering
    \includegraphics[width=0.7\linewidth]{lotka-volterra/pmh_cauchy_c2}
    \caption{Particle filter-based inference of the parameter $c_2$ in the Lotka-Volterra model. Uses Cauchy noise and a Gaussian observation model. The true value is shown in red.}
    \label{fig:lv-pmh-cauchy-c2}
\end{figure}

\begin{figure}[ht]
    \centering
    \includegraphics[width=0.7\linewidth]{lotka-volterra/pmh_cauchy_c3}
    \caption{Particle filter-based inference of the parameter $c_3$ in the Lotka-Volterra model. Uses Cauchy noise and a Gaussian observation model. The true value is shown in red.}
    \label{fig:lv-pmh-cauchy-c3}
\end{figure}

Clearly, Cauchy noise corrupts the sequence too much, and the results are miserable. The accepted parameters are almost constant, and the posterior distribution is not even remotely-well approximated. This is an expected behavior, since the particle filter is known not to perform well under model misspecification.



\subsection{Inference using ABC}
In this section, we apply \autoref{alg:marginal-metropolis-hastings-abc}, the variant of the Metropolis-Hastings algorithm depending on ABC.

\paragraph{Gaussian noise, Gaussian kernel}
At first, we again corrupt the sequence $\by_t$ by a Gaussian noise, as indicated above. We then run \autoref{alg:marginal-metropolis-hastings-abc} with a Gaussian kernel to infer about the parameters $\btheta$. The results are shown in \autoref{fig:lv-abcmh-gauss-gauss-c1}, \autoref{fig:lv-abcmh-gauss-gauss-c2} and \autoref{fig:lv-abcmh-gauss-gauss-c3}.

\begin{figure}[ht]
    \centering
    \includegraphics[width=0.7\linewidth]{lotka-volterra/abcmh_gauss_gauss_c1}
    \caption{ABC-based inference of the parameter $c_1$ in the Lotka-Volterra model. Uses Gaussian noise and a Gaussian kernel. The true value is shown in red.}
    \label{fig:lv-abcmh-gauss-gauss-c1}
\end{figure}

\begin{figure}[ht]
    \centering
    \includegraphics[width=0.7\linewidth]{lotka-volterra/abcmh_gauss_gauss_c2}
    \caption{ABC-based inference of the parameter $c_2$ in the Lotka-Volterra model. Uses Gaussian noise and a Gaussian kernel. The true value is shown in red.}
    \label{fig:lv-abcmh-gauss-gauss-c2}
\end{figure}

\begin{figure}[ht]
    \centering
    \includegraphics[width=0.7\linewidth]{lotka-volterra/abcmh_gauss_gauss_c3}
    \caption{ABC-based inference of the parameter $c_3$ in the Lotka-Volterra model. Uses Gaussian noise and a Gaussian kernel. The true value is shown in red.}
    \label{fig:lv-abcmh-gauss-gauss-c3}
\end{figure}

Compared to the results obtained by running the particle filter-based inference with a correct observation model, the results are slightly worse in the case of $c_1$, as the true value is in a region with a lower probability. Somewhat worse result is to be expected though, since the ABC methods provide only an approximation. Otherwise, the results are comparable to those utilizing the particle filter.


\paragraph{Cauchy noise, Gaussian kernel}
Next, we again corrupt the observation sequence by the heavy-tailed Cauchy noise. First, we keep the Gaussian kernel to calculate the importance weights. The results are in \autoref{fig:lv-abcmh-cauchy-gauss-c1}, \autoref{fig:lv-abcmh-cauchy-gauss-c2} and \autoref{fig:lv-abcmh-cauchy-gauss-c3}.

Compared to the particle filter with a misspecified observation model, the filter does not collapse at all. Instead, it remains stable, and the results resemble those obtained from a particle filter assuming a correct observation model, or those given by the previous ABC use-case.

This shows the strength of the ABC approximation -- even under a heavy-tailed noise such as the Cauchy one, using a set of simulated pseudo-observations with a suitable kernel function allows the likelihood estimate to remain stable even under a severe model misspecification.

\begin{figure}[ht]
    \centering
    \includegraphics[width=0.7\linewidth]{lotka-volterra/abcmh_cauchy_gauss_c1}
    \caption{ABC-based inference of the parameter $c_1$ in the Lotka-Volterra model. Uses Cauchy noise and a Gaussian kernel. The true value is shown in red.}
    \label{fig:lv-abcmh-cauchy-gauss-c1}
\end{figure}

\begin{figure}[ht]
    \centering
    \includegraphics[width=0.7\linewidth]{lotka-volterra/abcmh_cauchy_gauss_c2}
    \caption{ABC-based inference of the parameter $c_2$ in the Lotka-Volterra model. Uses Cauchy noise and a Gaussian kernel. The true value is shown in red.}
    \label{fig:lv-abcmh-cauchy-gauss-c2}
\end{figure}

\begin{figure}[ht]
    \centering
    \includegraphics[width=0.7\linewidth]{lotka-volterra/abcmh_cauchy_gauss_c3}
    \caption{ABC-based inference of the parameter $c_3$ in the Lotka-Volterra model. Uses Cauchy noise and a Gaussian kernel. The true value is shown in red.}
    \label{fig:lv-abcmh-cauchy-gauss-c3}
\end{figure}

\paragraph{Cauchy noise, Cauchy kernel}
Finally, we repeat the same experiment as in the previous section, but use a Cauchy kernel instead of the Gaussian one. The results are very similar to those obtained using a Gaussian kernel, indicating that the filter is fairly robust to kernel choice. This agrees with the conclusion provided by \cite{dedecius}.

The Cauchy kernel might be preferable to the Gaussian one for computational reasons -- its quantile function (required for kernel width tuning) can be calculated without resorting to numerical approximations.

\begin{figure}[ht]
    \centering
    \includegraphics[width=0.7\linewidth]{lotka-volterra/abcmh_cauchy_cauchy_c1}
    \caption{ABC-based inference of the parameter $c_1$ in the Lotka-Volterra model. Uses Cauchy noise and a Cauchy kernel. The true value is shown in red.}
    \label{fig:lv-abcmh-cauchy-cauchy-c1}
\end{figure}

\begin{figure}[ht]
    \centering
    \includegraphics[width=0.7\linewidth]{lotka-volterra/abcmh_cauchy_cauchy_c2}
    \caption{ABC-based inference of the parameter $c_2$ in the Lotka-Volterra model. Uses Cauchy noise and a Cauchy kernel. The true value is shown in red.}
    \label{fig:lv-abcmh-cauchy-cauchy-c2}
\end{figure}

\begin{figure}[ht]
    \centering
    \includegraphics[width=0.7\linewidth]{lotka-volterra/abcmh_cauchy_cauchy_c3}
    \caption{ABC-based inference of the parameter $c_3$ in the Lotka-Volterra model. Uses Cauchy noise and a Cauchy kernel. The true value is shown in red.}
    \label{fig:lv-abcmh-cauchy-cauchy-c3}
\end{figure}



\section{Prokaryotic auto-regulation model} \label{sec:autoregulation}
\subsection{Problem description}

\subsection{Inference using the particle filter}

\subsection{Inference using ABC}
\chapter{Conclusion and future work}
\label{chap:conclusion}


% Bibliography
\bibliographystyle{abbrvnat}
\bibliography{tex/references}


% Appendices
\appendix
\chapter{Attached files}
\label{chap:appendix}

The attached files contain the source codes and input data necessary to run the experiments. The main scripts used to run the simulations come with a command line interface containing a help message listing the accepted arguments. The attachment structure is specified in \autoref{tab:attached-files}.

\begin{table}[ht]
    \centering
    \begin{tabular}{ll}
        \hline
        \bf File & \bf Description \\ \hline
        \texttt{/data} & Serialized input data for the two experiments. \\
        \texttt{auto\_regulation.py} & Script to run the autoregulation experiment. \\
        \texttt{auto\_regulation\_routines.pyx} & Additional routines writen in Cython. \\
        \texttt{lotka\_volterra.py} & Script to run the Lotka-Volterra experiment. \\
        \texttt{lotka\_volterra\_routines.pyx} & Additional routines written in Cython. \\
        \texttt{mcmc.py} & Implementation of the inference methods. \\
        \texttt{utils.py} & Miscellaneous utility functions. \\
        \hline
    \end{tabular}
    \caption{List of attached files.}
    \label{tab:attached-files}
\end{table}


\end{document}